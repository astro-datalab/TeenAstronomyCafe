
% Default to the notebook output style

    


% Inherit from the specified cell style.




    
\documentclass[11pt]{article}

    
    
    \usepackage[T1]{fontenc}
    % Nicer default font (+ math font) than Computer Modern for most use cases
    \usepackage{mathpazo}

    % Basic figure setup, for now with no caption control since it's done
    % automatically by Pandoc (which extracts ![](path) syntax from Markdown).
    \usepackage{graphicx}
    % We will generate all images so they have a width \maxwidth. This means
    % that they will get their normal width if they fit onto the page, but
    % are scaled down if they would overflow the margins.
    \makeatletter
    \def\maxwidth{\ifdim\Gin@nat@width>\linewidth\linewidth
    \else\Gin@nat@width\fi}
    \makeatother
    \let\Oldincludegraphics\includegraphics
    % Set max figure width to be 80% of text width, for now hardcoded.
    \renewcommand{\includegraphics}[1]{\Oldincludegraphics[width=.8\maxwidth]{#1}}
    % Ensure that by default, figures have no caption (until we provide a
    % proper Figure object with a Caption API and a way to capture that
    % in the conversion process - todo).
    \usepackage{caption}
    \DeclareCaptionLabelFormat{nolabel}{}
    \captionsetup{labelformat=nolabel}

    \usepackage{adjustbox} % Used to constrain images to a maximum size 
    \usepackage{xcolor} % Allow colors to be defined
    \usepackage{enumerate} % Needed for markdown enumerations to work
    \usepackage{geometry} % Used to adjust the document margins
    \usepackage{amsmath} % Equations
    \usepackage{amssymb} % Equations
    \usepackage{textcomp} % defines textquotesingle
    % Hack from http://tex.stackexchange.com/a/47451/13684:
    \AtBeginDocument{%
        \def\PYZsq{\textquotesingle}% Upright quotes in Pygmentized code
    }
    \usepackage{upquote} % Upright quotes for verbatim code
    \usepackage{eurosym} % defines \euro
    \usepackage[mathletters]{ucs} % Extended unicode (utf-8) support
    \usepackage[utf8x]{inputenc} % Allow utf-8 characters in the tex document
    \usepackage{fancyvrb} % verbatim replacement that allows latex
    \usepackage{grffile} % extends the file name processing of package graphics 
                         % to support a larger range 
    % The hyperref package gives us a pdf with properly built
    % internal navigation ('pdf bookmarks' for the table of contents,
    % internal cross-reference links, web links for URLs, etc.)
    \usepackage{hyperref}
    \usepackage{longtable} % longtable support required by pandoc >1.10
    \usepackage{booktabs}  % table support for pandoc > 1.12.2
    \usepackage[inline]{enumitem} % IRkernel/repr support (it uses the enumerate* environment)
    \usepackage[normalem]{ulem} % ulem is needed to support strikethroughs (\sout)
                                % normalem makes italics be italics, not underlines
    

    
    
    % Colors for the hyperref package
    \definecolor{urlcolor}{rgb}{0,.145,.698}
    \definecolor{linkcolor}{rgb}{.71,0.21,0.01}
    \definecolor{citecolor}{rgb}{.12,.54,.11}

    % ANSI colors
    \definecolor{ansi-black}{HTML}{3E424D}
    \definecolor{ansi-black-intense}{HTML}{282C36}
    \definecolor{ansi-red}{HTML}{E75C58}
    \definecolor{ansi-red-intense}{HTML}{B22B31}
    \definecolor{ansi-green}{HTML}{00A250}
    \definecolor{ansi-green-intense}{HTML}{007427}
    \definecolor{ansi-yellow}{HTML}{DDB62B}
    \definecolor{ansi-yellow-intense}{HTML}{B27D12}
    \definecolor{ansi-blue}{HTML}{208FFB}
    \definecolor{ansi-blue-intense}{HTML}{0065CA}
    \definecolor{ansi-magenta}{HTML}{D160C4}
    \definecolor{ansi-magenta-intense}{HTML}{A03196}
    \definecolor{ansi-cyan}{HTML}{60C6C8}
    \definecolor{ansi-cyan-intense}{HTML}{258F8F}
    \definecolor{ansi-white}{HTML}{C5C1B4}
    \definecolor{ansi-white-intense}{HTML}{A1A6B2}

    % commands and environments needed by pandoc snippets
    % extracted from the output of `pandoc -s`
    \providecommand{\tightlist}{%
      \setlength{\itemsep}{0pt}\setlength{\parskip}{0pt}}
    \DefineVerbatimEnvironment{Highlighting}{Verbatim}{commandchars=\\\{\}}
    % Add ',fontsize=\small' for more characters per line
    \newenvironment{Shaded}{}{}
    \newcommand{\KeywordTok}[1]{\textcolor[rgb]{0.00,0.44,0.13}{\textbf{{#1}}}}
    \newcommand{\DataTypeTok}[1]{\textcolor[rgb]{0.56,0.13,0.00}{{#1}}}
    \newcommand{\DecValTok}[1]{\textcolor[rgb]{0.25,0.63,0.44}{{#1}}}
    \newcommand{\BaseNTok}[1]{\textcolor[rgb]{0.25,0.63,0.44}{{#1}}}
    \newcommand{\FloatTok}[1]{\textcolor[rgb]{0.25,0.63,0.44}{{#1}}}
    \newcommand{\CharTok}[1]{\textcolor[rgb]{0.25,0.44,0.63}{{#1}}}
    \newcommand{\StringTok}[1]{\textcolor[rgb]{0.25,0.44,0.63}{{#1}}}
    \newcommand{\CommentTok}[1]{\textcolor[rgb]{0.38,0.63,0.69}{\textit{{#1}}}}
    \newcommand{\OtherTok}[1]{\textcolor[rgb]{0.00,0.44,0.13}{{#1}}}
    \newcommand{\AlertTok}[1]{\textcolor[rgb]{1.00,0.00,0.00}{\textbf{{#1}}}}
    \newcommand{\FunctionTok}[1]{\textcolor[rgb]{0.02,0.16,0.49}{{#1}}}
    \newcommand{\RegionMarkerTok}[1]{{#1}}
    \newcommand{\ErrorTok}[1]{\textcolor[rgb]{1.00,0.00,0.00}{\textbf{{#1}}}}
    \newcommand{\NormalTok}[1]{{#1}}
    
    % Additional commands for more recent versions of Pandoc
    \newcommand{\ConstantTok}[1]{\textcolor[rgb]{0.53,0.00,0.00}{{#1}}}
    \newcommand{\SpecialCharTok}[1]{\textcolor[rgb]{0.25,0.44,0.63}{{#1}}}
    \newcommand{\VerbatimStringTok}[1]{\textcolor[rgb]{0.25,0.44,0.63}{{#1}}}
    \newcommand{\SpecialStringTok}[1]{\textcolor[rgb]{0.73,0.40,0.53}{{#1}}}
    \newcommand{\ImportTok}[1]{{#1}}
    \newcommand{\DocumentationTok}[1]{\textcolor[rgb]{0.73,0.13,0.13}{\textit{{#1}}}}
    \newcommand{\AnnotationTok}[1]{\textcolor[rgb]{0.38,0.63,0.69}{\textbf{\textit{{#1}}}}}
    \newcommand{\CommentVarTok}[1]{\textcolor[rgb]{0.38,0.63,0.69}{\textbf{\textit{{#1}}}}}
    \newcommand{\VariableTok}[1]{\textcolor[rgb]{0.10,0.09,0.49}{{#1}}}
    \newcommand{\ControlFlowTok}[1]{\textcolor[rgb]{0.00,0.44,0.13}{\textbf{{#1}}}}
    \newcommand{\OperatorTok}[1]{\textcolor[rgb]{0.40,0.40,0.40}{{#1}}}
    \newcommand{\BuiltInTok}[1]{{#1}}
    \newcommand{\ExtensionTok}[1]{{#1}}
    \newcommand{\PreprocessorTok}[1]{\textcolor[rgb]{0.74,0.48,0.00}{{#1}}}
    \newcommand{\AttributeTok}[1]{\textcolor[rgb]{0.49,0.56,0.16}{{#1}}}
    \newcommand{\InformationTok}[1]{\textcolor[rgb]{0.38,0.63,0.69}{\textbf{\textit{{#1}}}}}
    \newcommand{\WarningTok}[1]{\textcolor[rgb]{0.38,0.63,0.69}{\textbf{\textit{{#1}}}}}
    
    
    % Define a nice break command that doesn't care if a line doesn't already
    % exist.
    \def\br{\hspace*{\fill} \\* }
    % Math Jax compatability definitions
    \def\gt{>}
    \def\lt{<}
    % Document parameters
    \title{Large\_Scale\_Structure\_of\_the\_Universe}
    
    
    

    % Pygments definitions
    
\makeatletter
\def\PY@reset{\let\PY@it=\relax \let\PY@bf=\relax%
    \let\PY@ul=\relax \let\PY@tc=\relax%
    \let\PY@bc=\relax \let\PY@ff=\relax}
\def\PY@tok#1{\csname PY@tok@#1\endcsname}
\def\PY@toks#1+{\ifx\relax#1\empty\else%
    \PY@tok{#1}\expandafter\PY@toks\fi}
\def\PY@do#1{\PY@bc{\PY@tc{\PY@ul{%
    \PY@it{\PY@bf{\PY@ff{#1}}}}}}}
\def\PY#1#2{\PY@reset\PY@toks#1+\relax+\PY@do{#2}}

\expandafter\def\csname PY@tok@w\endcsname{\def\PY@tc##1{\textcolor[rgb]{0.73,0.73,0.73}{##1}}}
\expandafter\def\csname PY@tok@c\endcsname{\let\PY@it=\textit\def\PY@tc##1{\textcolor[rgb]{0.25,0.50,0.50}{##1}}}
\expandafter\def\csname PY@tok@cp\endcsname{\def\PY@tc##1{\textcolor[rgb]{0.74,0.48,0.00}{##1}}}
\expandafter\def\csname PY@tok@k\endcsname{\let\PY@bf=\textbf\def\PY@tc##1{\textcolor[rgb]{0.00,0.50,0.00}{##1}}}
\expandafter\def\csname PY@tok@kp\endcsname{\def\PY@tc##1{\textcolor[rgb]{0.00,0.50,0.00}{##1}}}
\expandafter\def\csname PY@tok@kt\endcsname{\def\PY@tc##1{\textcolor[rgb]{0.69,0.00,0.25}{##1}}}
\expandafter\def\csname PY@tok@o\endcsname{\def\PY@tc##1{\textcolor[rgb]{0.40,0.40,0.40}{##1}}}
\expandafter\def\csname PY@tok@ow\endcsname{\let\PY@bf=\textbf\def\PY@tc##1{\textcolor[rgb]{0.67,0.13,1.00}{##1}}}
\expandafter\def\csname PY@tok@nb\endcsname{\def\PY@tc##1{\textcolor[rgb]{0.00,0.50,0.00}{##1}}}
\expandafter\def\csname PY@tok@nf\endcsname{\def\PY@tc##1{\textcolor[rgb]{0.00,0.00,1.00}{##1}}}
\expandafter\def\csname PY@tok@nc\endcsname{\let\PY@bf=\textbf\def\PY@tc##1{\textcolor[rgb]{0.00,0.00,1.00}{##1}}}
\expandafter\def\csname PY@tok@nn\endcsname{\let\PY@bf=\textbf\def\PY@tc##1{\textcolor[rgb]{0.00,0.00,1.00}{##1}}}
\expandafter\def\csname PY@tok@ne\endcsname{\let\PY@bf=\textbf\def\PY@tc##1{\textcolor[rgb]{0.82,0.25,0.23}{##1}}}
\expandafter\def\csname PY@tok@nv\endcsname{\def\PY@tc##1{\textcolor[rgb]{0.10,0.09,0.49}{##1}}}
\expandafter\def\csname PY@tok@no\endcsname{\def\PY@tc##1{\textcolor[rgb]{0.53,0.00,0.00}{##1}}}
\expandafter\def\csname PY@tok@nl\endcsname{\def\PY@tc##1{\textcolor[rgb]{0.63,0.63,0.00}{##1}}}
\expandafter\def\csname PY@tok@ni\endcsname{\let\PY@bf=\textbf\def\PY@tc##1{\textcolor[rgb]{0.60,0.60,0.60}{##1}}}
\expandafter\def\csname PY@tok@na\endcsname{\def\PY@tc##1{\textcolor[rgb]{0.49,0.56,0.16}{##1}}}
\expandafter\def\csname PY@tok@nt\endcsname{\let\PY@bf=\textbf\def\PY@tc##1{\textcolor[rgb]{0.00,0.50,0.00}{##1}}}
\expandafter\def\csname PY@tok@nd\endcsname{\def\PY@tc##1{\textcolor[rgb]{0.67,0.13,1.00}{##1}}}
\expandafter\def\csname PY@tok@s\endcsname{\def\PY@tc##1{\textcolor[rgb]{0.73,0.13,0.13}{##1}}}
\expandafter\def\csname PY@tok@sd\endcsname{\let\PY@it=\textit\def\PY@tc##1{\textcolor[rgb]{0.73,0.13,0.13}{##1}}}
\expandafter\def\csname PY@tok@si\endcsname{\let\PY@bf=\textbf\def\PY@tc##1{\textcolor[rgb]{0.73,0.40,0.53}{##1}}}
\expandafter\def\csname PY@tok@se\endcsname{\let\PY@bf=\textbf\def\PY@tc##1{\textcolor[rgb]{0.73,0.40,0.13}{##1}}}
\expandafter\def\csname PY@tok@sr\endcsname{\def\PY@tc##1{\textcolor[rgb]{0.73,0.40,0.53}{##1}}}
\expandafter\def\csname PY@tok@ss\endcsname{\def\PY@tc##1{\textcolor[rgb]{0.10,0.09,0.49}{##1}}}
\expandafter\def\csname PY@tok@sx\endcsname{\def\PY@tc##1{\textcolor[rgb]{0.00,0.50,0.00}{##1}}}
\expandafter\def\csname PY@tok@m\endcsname{\def\PY@tc##1{\textcolor[rgb]{0.40,0.40,0.40}{##1}}}
\expandafter\def\csname PY@tok@gh\endcsname{\let\PY@bf=\textbf\def\PY@tc##1{\textcolor[rgb]{0.00,0.00,0.50}{##1}}}
\expandafter\def\csname PY@tok@gu\endcsname{\let\PY@bf=\textbf\def\PY@tc##1{\textcolor[rgb]{0.50,0.00,0.50}{##1}}}
\expandafter\def\csname PY@tok@gd\endcsname{\def\PY@tc##1{\textcolor[rgb]{0.63,0.00,0.00}{##1}}}
\expandafter\def\csname PY@tok@gi\endcsname{\def\PY@tc##1{\textcolor[rgb]{0.00,0.63,0.00}{##1}}}
\expandafter\def\csname PY@tok@gr\endcsname{\def\PY@tc##1{\textcolor[rgb]{1.00,0.00,0.00}{##1}}}
\expandafter\def\csname PY@tok@ge\endcsname{\let\PY@it=\textit}
\expandafter\def\csname PY@tok@gs\endcsname{\let\PY@bf=\textbf}
\expandafter\def\csname PY@tok@gp\endcsname{\let\PY@bf=\textbf\def\PY@tc##1{\textcolor[rgb]{0.00,0.00,0.50}{##1}}}
\expandafter\def\csname PY@tok@go\endcsname{\def\PY@tc##1{\textcolor[rgb]{0.53,0.53,0.53}{##1}}}
\expandafter\def\csname PY@tok@gt\endcsname{\def\PY@tc##1{\textcolor[rgb]{0.00,0.27,0.87}{##1}}}
\expandafter\def\csname PY@tok@err\endcsname{\def\PY@bc##1{\setlength{\fboxsep}{0pt}\fcolorbox[rgb]{1.00,0.00,0.00}{1,1,1}{\strut ##1}}}
\expandafter\def\csname PY@tok@kc\endcsname{\let\PY@bf=\textbf\def\PY@tc##1{\textcolor[rgb]{0.00,0.50,0.00}{##1}}}
\expandafter\def\csname PY@tok@kd\endcsname{\let\PY@bf=\textbf\def\PY@tc##1{\textcolor[rgb]{0.00,0.50,0.00}{##1}}}
\expandafter\def\csname PY@tok@kn\endcsname{\let\PY@bf=\textbf\def\PY@tc##1{\textcolor[rgb]{0.00,0.50,0.00}{##1}}}
\expandafter\def\csname PY@tok@kr\endcsname{\let\PY@bf=\textbf\def\PY@tc##1{\textcolor[rgb]{0.00,0.50,0.00}{##1}}}
\expandafter\def\csname PY@tok@bp\endcsname{\def\PY@tc##1{\textcolor[rgb]{0.00,0.50,0.00}{##1}}}
\expandafter\def\csname PY@tok@fm\endcsname{\def\PY@tc##1{\textcolor[rgb]{0.00,0.00,1.00}{##1}}}
\expandafter\def\csname PY@tok@vc\endcsname{\def\PY@tc##1{\textcolor[rgb]{0.10,0.09,0.49}{##1}}}
\expandafter\def\csname PY@tok@vg\endcsname{\def\PY@tc##1{\textcolor[rgb]{0.10,0.09,0.49}{##1}}}
\expandafter\def\csname PY@tok@vi\endcsname{\def\PY@tc##1{\textcolor[rgb]{0.10,0.09,0.49}{##1}}}
\expandafter\def\csname PY@tok@vm\endcsname{\def\PY@tc##1{\textcolor[rgb]{0.10,0.09,0.49}{##1}}}
\expandafter\def\csname PY@tok@sa\endcsname{\def\PY@tc##1{\textcolor[rgb]{0.73,0.13,0.13}{##1}}}
\expandafter\def\csname PY@tok@sb\endcsname{\def\PY@tc##1{\textcolor[rgb]{0.73,0.13,0.13}{##1}}}
\expandafter\def\csname PY@tok@sc\endcsname{\def\PY@tc##1{\textcolor[rgb]{0.73,0.13,0.13}{##1}}}
\expandafter\def\csname PY@tok@dl\endcsname{\def\PY@tc##1{\textcolor[rgb]{0.73,0.13,0.13}{##1}}}
\expandafter\def\csname PY@tok@s2\endcsname{\def\PY@tc##1{\textcolor[rgb]{0.73,0.13,0.13}{##1}}}
\expandafter\def\csname PY@tok@sh\endcsname{\def\PY@tc##1{\textcolor[rgb]{0.73,0.13,0.13}{##1}}}
\expandafter\def\csname PY@tok@s1\endcsname{\def\PY@tc##1{\textcolor[rgb]{0.73,0.13,0.13}{##1}}}
\expandafter\def\csname PY@tok@mb\endcsname{\def\PY@tc##1{\textcolor[rgb]{0.40,0.40,0.40}{##1}}}
\expandafter\def\csname PY@tok@mf\endcsname{\def\PY@tc##1{\textcolor[rgb]{0.40,0.40,0.40}{##1}}}
\expandafter\def\csname PY@tok@mh\endcsname{\def\PY@tc##1{\textcolor[rgb]{0.40,0.40,0.40}{##1}}}
\expandafter\def\csname PY@tok@mi\endcsname{\def\PY@tc##1{\textcolor[rgb]{0.40,0.40,0.40}{##1}}}
\expandafter\def\csname PY@tok@il\endcsname{\def\PY@tc##1{\textcolor[rgb]{0.40,0.40,0.40}{##1}}}
\expandafter\def\csname PY@tok@mo\endcsname{\def\PY@tc##1{\textcolor[rgb]{0.40,0.40,0.40}{##1}}}
\expandafter\def\csname PY@tok@ch\endcsname{\let\PY@it=\textit\def\PY@tc##1{\textcolor[rgb]{0.25,0.50,0.50}{##1}}}
\expandafter\def\csname PY@tok@cm\endcsname{\let\PY@it=\textit\def\PY@tc##1{\textcolor[rgb]{0.25,0.50,0.50}{##1}}}
\expandafter\def\csname PY@tok@cpf\endcsname{\let\PY@it=\textit\def\PY@tc##1{\textcolor[rgb]{0.25,0.50,0.50}{##1}}}
\expandafter\def\csname PY@tok@c1\endcsname{\let\PY@it=\textit\def\PY@tc##1{\textcolor[rgb]{0.25,0.50,0.50}{##1}}}
\expandafter\def\csname PY@tok@cs\endcsname{\let\PY@it=\textit\def\PY@tc##1{\textcolor[rgb]{0.25,0.50,0.50}{##1}}}

\def\PYZbs{\char`\\}
\def\PYZus{\char`\_}
\def\PYZob{\char`\{}
\def\PYZcb{\char`\}}
\def\PYZca{\char`\^}
\def\PYZam{\char`\&}
\def\PYZlt{\char`\<}
\def\PYZgt{\char`\>}
\def\PYZsh{\char`\#}
\def\PYZpc{\char`\%}
\def\PYZdl{\char`\$}
\def\PYZhy{\char`\-}
\def\PYZsq{\char`\'}
\def\PYZdq{\char`\"}
\def\PYZti{\char`\~}
% for compatibility with earlier versions
\def\PYZat{@}
\def\PYZlb{[}
\def\PYZrb{]}
\makeatother


    % Exact colors from NB
    \definecolor{incolor}{rgb}{0.0, 0.0, 0.5}
    \definecolor{outcolor}{rgb}{0.545, 0.0, 0.0}



    
    % Prevent overflowing lines due to hard-to-break entities
    \sloppy 
    % Setup hyperref package
    \hypersetup{
      breaklinks=true,  % so long urls are correctly broken across lines
      colorlinks=true,
      urlcolor=urlcolor,
      linkcolor=linkcolor,
      citecolor=citecolor,
      }
    % Slightly bigger margins than the latex defaults
    
    \geometry{verbose,tmargin=1in,bmargin=1in,lmargin=1in,rmargin=1in}
    
    

    \begin{document}
    
    
    \maketitle
    
    

    
    \section{\texorpdfstring{\textbf{Our Vast Universe Probed with Big
Data}}{Our Vast Universe Probed with Big Data}}\label{our-vast-universe-probed-with-big-data}

\paragraph{\texorpdfstring{Written by Stephanie Juneau, NOAO
(\url{mailto:sjuneau@noao.edu}), with contributions from Leah Fulmer \&
Gautham
Narayan}{Written by Stephanie Juneau, NOAO (mailto:sjuneau@noao.edu), with contributions from Leah Fulmer \& Gautham Narayan}}\label{written-by-stephanie-juneau-noao-mailtosjuneaunoao.edu-with-contributions-from-leah-fulmer-gautham-narayan}

Last updated on 10/12/2018

 This notebook is interactive, and we will use it to learn and explore
the distribution of galaxies in the universe. We follow a similar
approach as other notebooks you might have used, in particular the one
written by Gautham Narayan.

Feel comfortable to ask questions as you go along!

    \subsection{BEGIN HERE: How to use this
notebook}\label{begin-here-how-to-use-this-notebook}

The webpage you are in is actually an app - much like you'd run on your
cellphone. This app consists of cells.

Each "input" cell (something with an "In" to the left) contains code -
instructions to make the computer do something.

To activate or select a cell, you first need to click on it.

You \textbf{execute a cell with Shift+Enter} on the keyboard - this
makes the computer execute your instructions. That's what this app does!

You can \textbf{modify the code by typing into the cell} and then
execute again the new code with Shift+Enter.

You can try it for yourself at https://try.jupyter.org/

    \section{Activity 1: How Far Away are
Galaxies?}\label{activity-1-how-far-away-are-galaxies}

    In this activity, you will learn how astronomers measure distances to
galaxies. You will get to compare galaxies to figure out which ones are
closer or further away from us. You can then use this method for many
more galaxies on your own as well!

    \begin{Verbatim}[commandchars=\\\{\}]
{\color{incolor}In [{\color{incolor} }]:} \PY{c+c1}{\PYZsh{} Ignore this stuff \PYZhy{} it is to setup the plotting environment in your browser}
        \PY{c+c1}{\PYZsh{} Just hit Shift + Enter here, and move on}
        \PY{o}{\PYZpc{}}\PY{k}{matplotlib} notebook
        \PY{o}{\PYZpc{}}\PY{k}{pylab}
        \PY{k+kn}{import} \PY{n+nn}{matplotlib}\PY{n+nn}{.}\PY{n+nn}{path} \PY{k}{as} \PY{n+nn}{mpath}
        \PY{k+kn}{from} \PY{n+nn}{astroML}\PY{n+nn}{.}\PY{n+nn}{datasets} \PY{k}{import} \PY{n}{fetch\PYZus{}sdss\PYZus{}spectrum}\PY{p}{,} \PY{n}{fetch\PYZus{}vega\PYZus{}spectrum}\PY{p}{,} \PY{n}{fetch\PYZus{}sdss\PYZus{}S82standards}
        \PY{k+kn}{from} \PY{n+nn}{astroML}\PY{n+nn}{.}\PY{n+nn}{plotting} \PY{k}{import} \PY{n}{MultiAxes}
        \PY{k+kn}{from} \PY{n+nn}{IPython}\PY{n+nn}{.}\PY{n+nn}{core}\PY{n+nn}{.}\PY{n+nn}{display} \PY{k}{import} \PY{n}{Image}\PY{p}{,} \PY{n}{display}
\end{Verbatim}


    As in the first notebook, we are going to use data from the
\textbf{Sloan Digital Sky Survey (SDSS)}. This project used a telescope
at Apache Point in New Mexico to look at the northern sky.

The Sloan Telescope at Apache Point, New Mexico. Image Credit: SDSS
Team, Fermilab Visual Media Services.

The Sloan survey team found millions of stars and galaxies, and made
their big data set public. In this activity, we will retrieve and
examine galaxy data!

So how did Sloan take spectra of millions of stars and galaxies? The
team used metal plates like the one shown below, with a hundreds of
holes aligned with the stars and galaxies to be observed. An optical
fiber is placed in each hole in order to transfer the light to the
instrument and camera. As you will see below, the data are identified by
their Plate number, their Fiber number, and the date when they were
obtained - the MJD (Modified Julian Date).

Holes in aluminum plates let the light from stars and galaxies passed to
an optical fiber to the instrument. Image credit: D. Long, SDSS-III

David Schlegel, Principal Investigator of the BOSS survey (follow-up to
SDSS), holding one fiber plug plate.

There were thousands of plates used (\textasciitilde{}2500 for SDSS),
each with 640 fibers, which together gives 1.6 million spectra
(including galaxies, stars, and extra spectra on blank sky).

    \subsection{Step 1.1: Plot a Reference
Spectrum}\label{step-1.1-plot-a-reference-spectrum}

A reference spectrum means that it is at redshift zero (not moving
toward or away from us). In this case, the reference spectrum is that of
a single star.

    \begin{Verbatim}[commandchars=\\\{\}]
{\color{incolor}In [{\color{incolor} }]:} \PY{c+c1}{\PYZsh{} Fetch single spectrum \PYZhy{} Enter the same \PYZdq{}Plate\PYZdq{}, \PYZdq{}MJD\PYZdq{} and \PYZdq{}Fiber\PYZdq{} numbers here}
        \PY{c+c1}{\PYZsh{} Then hit Shift+Enter}
        \PY{n}{plate} \PY{o}{=} \PY{l+m+mi}{396}
        \PY{n}{mjd} \PY{o}{=} \PY{l+m+mi}{51816}
        \PY{n}{fiber} \PY{o}{=} \PY{l+m+mi}{605}
        \PY{n}{spec} \PY{o}{=} \PY{n}{fetch\PYZus{}sdss\PYZus{}spectrum}\PY{p}{(}\PY{n}{plate}\PY{p}{,} \PY{n}{mjd}\PY{p}{,} \PY{n}{fiber}\PY{p}{)}
        
        \PY{c+c1}{\PYZsh{} now, we can plot the reference spectrum (at redshift=0)}
        \PY{n}{figure}\PY{p}{(}\PY{p}{)}
        \PY{n}{plot}\PY{p}{(}\PY{n}{spec}\PY{o}{.}\PY{n}{wavelength}\PY{p}{(}\PY{p}{)}\PY{p}{,} \PY{n}{spec}\PY{o}{.}\PY{n}{spectrum}\PY{o}{/}\PY{n}{spec}\PY{o}{.}\PY{n}{spectrum}\PY{o}{.}\PY{n}{max}\PY{p}{(}\PY{p}{)}\PY{p}{,} \PY{n}{color}\PY{o}{=}\PY{l+s+s1}{\PYZsq{}}\PY{l+s+s1}{red}\PY{l+s+s1}{\PYZsq{}}\PY{p}{)}
        \PY{n}{plt}\PY{o}{.}\PY{n}{title}\PY{p}{(}\PY{l+s+s1}{\PYZsq{}}\PY{l+s+s1}{Reference Spectrum}\PY{l+s+s1}{\PYZsq{}}\PY{p}{)}
        \PY{n}{xlim}\PY{p}{(}\PY{l+m+mi}{3800}\PY{p}{,}\PY{l+m+mi}{6000}\PY{p}{)}
        \PY{n}{xlabel}\PY{p}{(}\PY{l+s+s1}{\PYZsq{}}\PY{l+s+s1}{Wavelength}\PY{l+s+s1}{\PYZsq{}}\PY{p}{)}
        \PY{n}{ylabel}\PY{p}{(}\PY{l+s+s1}{\PYZsq{}}\PY{l+s+s1}{Brightness}\PY{l+s+s1}{\PYZsq{}}\PY{p}{)}
\end{Verbatim}


    \subsection{Step 1.2: Plot a Galaxy
Spectrum}\label{step-1.2-plot-a-galaxy-spectrum}

Here, you will plot the spectrum of a galaxy. Notice if there are
similarities and differences in its shape and lines relative to the
reference spectrum.

    \begin{Verbatim}[commandchars=\\\{\}]
{\color{incolor}In [{\color{incolor} }]:} \PY{c+c1}{\PYZsh{} Fetch the first galaxy spectrum}
        \PY{c+c1}{\PYZsh{} Then hit Shift+Enter}
        \PY{n}{plate} \PY{o}{=} \PY{l+m+mi}{2434}
        \PY{n}{mjd} \PY{o}{=} \PY{l+m+mi}{53826}
        \PY{n}{fiber} \PY{o}{=} \PY{l+m+mi}{359}
        \PY{n}{spec1} \PY{o}{=} \PY{n}{fetch\PYZus{}sdss\PYZus{}spectrum}\PY{p}{(}\PY{n}{plate}\PY{p}{,} \PY{n}{mjd}\PY{p}{,} \PY{n}{fiber}\PY{p}{)}
        
        \PY{c+c1}{\PYZsh{} now, we can plot the reference spectrum (at redshift=0)}
        \PY{n}{figure}\PY{p}{(}\PY{p}{)}
        \PY{n}{plot}\PY{p}{(}\PY{n}{spec1}\PY{o}{.}\PY{n}{wavelength}\PY{p}{(}\PY{p}{)}\PY{p}{,} \PY{n}{spec1}\PY{o}{.}\PY{n}{spectrum}\PY{o}{/}\PY{l+m+mf}{40.}\PY{p}{,} \PY{n}{color}\PY{o}{=}\PY{l+s+s1}{\PYZsq{}}\PY{l+s+s1}{black}\PY{l+s+s1}{\PYZsq{}}\PY{p}{)}
        \PY{n}{plt}\PY{o}{.}\PY{n}{title}\PY{p}{(}\PY{l+s+s1}{\PYZsq{}}\PY{l+s+s1}{Galaxy A}\PY{l+s+s1}{\PYZsq{}}\PY{p}{)}
        \PY{n}{xlabel}\PY{p}{(}\PY{l+s+s1}{\PYZsq{}}\PY{l+s+s1}{Wavelength}\PY{l+s+s1}{\PYZsq{}}\PY{p}{)}
        \PY{n}{ylabel}\PY{p}{(}\PY{l+s+s1}{\PYZsq{}}\PY{l+s+s1}{Brightness}\PY{l+s+s1}{\PYZsq{}}\PY{p}{)}
        \PY{n}{xlim}\PY{p}{(}\PY{l+m+mi}{3800}\PY{p}{,}\PY{l+m+mi}{6000}\PY{p}{)}
        \PY{n}{ylim}\PY{p}{(}\PY{o}{\PYZhy{}}\PY{l+m+mf}{0.1}\PY{p}{,} \PY{l+m+mf}{1.2}\PY{p}{)}
\end{Verbatim}


    \subsection{Questions}\label{questions}

Do you notice differences between the shapes two spectra?

Do you notice similar patterns in the line features (dips)?

    \subsection{Step 1.3: Measure
Redshifts}\label{step-1.3-measure-redshifts}

The next step here is to overlay a reference spectrum (called a
template) onto the galaxy spectra from above.

\paragraph{Reminder: a galaxy is a collection of billion of stars, so
the shape of the spectrum is not identical to the reference spectrum of
a single star. But because the stars have the same elements, notice
similar "dips" in the
spectra.}\label{reminder-a-galaxy-is-a-collection-of-billion-of-stars-so-the-shape-of-the-spectrum-is-not-identical-to-the-reference-spectrum-of-a-single-star.-but-because-the-stars-have-the-same-elements-notice-similar-dips-in-the-spectra.}

    \begin{Verbatim}[commandchars=\\\{\}]
{\color{incolor}In [{\color{incolor} }]:} \PY{c+c1}{\PYZsh{} redshift value (0 for a star, and upward for distant galaxies e.g.: }
        \PY{c+c1}{\PYZsh{} z = 0.01, 0.02, 0.05, ... 0.1, 0.2, ... 1.0)}
        \PY{c+c1}{\PYZsh{} First run this cell with zero redshift, and then adjust the value.}
        \PY{n}{z1} \PY{o}{=} \PY{l+m+mf}{0.}
        
        \PY{c+c1}{\PYZsh{} now, we can plot the reference spectrum (at redshift=0)}
        \PY{n}{figure}\PY{p}{(}\PY{p}{)}
        \PY{n}{plot}\PY{p}{(}\PY{n}{spec1}\PY{o}{.}\PY{n}{wavelength}\PY{p}{(}\PY{p}{)}\PY{p}{,} \PY{n}{spec1}\PY{o}{.}\PY{n}{spectrum}\PY{o}{/}\PY{l+m+mf}{45.}\PY{p}{,} \PY{n}{color}\PY{o}{=}\PY{l+s+s1}{\PYZsq{}}\PY{l+s+s1}{black}\PY{l+s+s1}{\PYZsq{}}\PY{p}{)}
        \PY{n}{plot}\PY{p}{(}\PY{n}{spec}\PY{o}{.}\PY{n}{wavelength}\PY{p}{(}\PY{p}{)}\PY{o}{*}\PY{p}{(}\PY{l+m+mf}{1.}\PY{o}{+}\PY{n}{z1}\PY{p}{)}\PY{p}{,} \PY{n}{spec}\PY{o}{.}\PY{n}{spectrum}\PY{o}{/}\PY{n}{spec}\PY{o}{.}\PY{n}{spectrum}\PY{o}{.}\PY{n}{max}\PY{p}{(}\PY{p}{)}\PY{o}{/}\PY{l+m+mf}{1.5}\PY{p}{,} \PY{n}{color}\PY{o}{=}\PY{l+s+s1}{\PYZsq{}}\PY{l+s+s1}{red}\PY{l+s+s1}{\PYZsq{}}\PY{p}{)}
        \PY{n}{plt}\PY{o}{.}\PY{n}{title}\PY{p}{(}\PY{l+s+s1}{\PYZsq{}}\PY{l+s+s1}{Galaxy A}\PY{l+s+s1}{\PYZsq{}}\PY{p}{)}
        \PY{n}{xlabel}\PY{p}{(}\PY{l+s+s1}{\PYZsq{}}\PY{l+s+s1}{Wavelength}\PY{l+s+s1}{\PYZsq{}}\PY{p}{)}
        \PY{n}{ylabel}\PY{p}{(}\PY{l+s+s1}{\PYZsq{}}\PY{l+s+s1}{Brightness}\PY{l+s+s1}{\PYZsq{}}\PY{p}{)}
        \PY{n}{xlim}\PY{p}{(}\PY{l+m+mi}{3800}\PY{p}{,}\PY{l+m+mi}{6000}\PY{p}{)}
        \PY{n}{ylim}\PY{p}{(}\PY{o}{\PYZhy{}}\PY{l+m+mf}{0.1}\PY{p}{,} \PY{l+m+mf}{1.2}\PY{p}{)}
\end{Verbatim}


    \subsection{Question:}\label{question}

Do you notice how the galaxy spectrum is shifted with respect to the
reference spectrum?

This is what we saw as the "redshift" due to the expansion of the
universe, which causes galaxies to appear to recede away from us.

    \section{MYSTERY Galaxy}\label{mystery-galaxy}

Each group will receive the information to fetch a different spectrum of
a mystery galaxy. This information will be on a piece of paper and
contains PLATE, MJD, FIBER and NORM numbers. Enter these numbers in the
cell below to measure the redshift (like you did above) for a new
galaxy. If you do not get a piece of paper, simply use the numbers that
are pre-entered below.

    \begin{Verbatim}[commandchars=\\\{\}]
{\color{incolor}In [{\color{incolor} }]:} \PY{c+c1}{\PYZsh{} Now we do it again for a NEW galaxy!}
        
        \PY{c+c1}{\PYZsh{} Replace the plate, mjd, fiber and norm with the numbers you received, }
        \PY{c+c1}{\PYZsh{} Then hit Shift+Enter}
        
        \PY{c+c1}{\PYZsh{}\PYZsh{} Mystery galaxy}
        \PY{n}{plate} \PY{o}{=} \PY{l+m+mi}{2121}
        \PY{n}{mjd} \PY{o}{=} \PY{l+m+mi}{54180}
        \PY{n}{fiber} \PY{o}{=} \PY{l+m+mi}{414}
        \PY{n}{norm} \PY{o}{=} \PY{l+m+mf}{22.}
        
        \PY{c+c1}{\PYZsh{} Or choose one of the galaxies below (you need to delete the \PYZdq{}\PYZsh{}\PYZdq{} sign in front of *ONE* block}
        \PY{c+c1}{\PYZsh{} after the Galaxy name, in order to set plate, mjd, fiber and norm)}
        
        \PY{c+c1}{\PYZsh{}\PYZsh{} Galaxy Mystery2 \PYZhy{} detele the \PYZdq{}\PYZsh{}\PYZdq{} for the lines below }
        \PY{c+c1}{\PYZsh{}plate = 1759}
        \PY{c+c1}{\PYZsh{}mjd = 53081}
        \PY{c+c1}{\PYZsh{}fiber = 126}
        \PY{c+c1}{\PYZsh{}norm = 15.}
        
        \PY{c+c1}{\PYZsh{}\PYZsh{} Galaxy Mystery3 \PYZhy{} detele the \PYZdq{}\PYZsh{}\PYZdq{} for the lines below}
        \PY{c+c1}{\PYZsh{}plate = 1839}
        \PY{c+c1}{\PYZsh{}mjd = 53471}
        \PY{c+c1}{\PYZsh{}fiber = 310}
        \PY{c+c1}{\PYZsh{}norm = 20.}
        
        \PY{c+c1}{\PYZsh{}\PYZsh{} Galaxy Mystery4 \PYZhy{} detele the \PYZdq{}\PYZsh{}\PYZdq{} for the lines below}
        \PY{c+c1}{\PYZsh{}plate = 2121}
        \PY{c+c1}{\PYZsh{}mjd = 54180}
        \PY{c+c1}{\PYZsh{}fiber = 523}
        \PY{c+c1}{\PYZsh{}norm = 350.}
        
        \PY{n}{spec2} \PY{o}{=} \PY{n}{fetch\PYZus{}sdss\PYZus{}spectrum}\PY{p}{(}\PY{n}{plate}\PY{p}{,} \PY{n}{mjd}\PY{p}{,} \PY{n}{fiber}\PY{p}{)}
        
        \PY{c+c1}{\PYZsh{} redshift value (0 for a star, and upward for distant galaxies e.g.: }
        \PY{c+c1}{\PYZsh{} z=0.01, 0.02, 0.05, ... 0.1, 0.2, 0.3, ... 1.0)}
        \PY{c+c1}{\PYZsh{} First run this cell with zero redshift, and then adjust the value.}
        \PY{n}{z2} \PY{o}{=} \PY{l+m+mf}{0.0}
        
        \PY{c+c1}{\PYZsh{} now, we can plot the reference spectrum (at redshift=0)}
        \PY{n}{figure}\PY{p}{(}\PY{p}{)}
        \PY{n}{plot}\PY{p}{(}\PY{n}{spec2}\PY{o}{.}\PY{n}{wavelength}\PY{p}{(}\PY{p}{)}\PY{p}{,} \PY{n}{spec2}\PY{o}{.}\PY{n}{spectrum}\PY{o}{/}\PY{n}{norm}\PY{p}{,} \PY{n}{color}\PY{o}{=}\PY{l+s+s1}{\PYZsq{}}\PY{l+s+s1}{black}\PY{l+s+s1}{\PYZsq{}}\PY{p}{)}
        \PY{n}{plot}\PY{p}{(}\PY{n}{spec}\PY{o}{.}\PY{n}{wavelength}\PY{p}{(}\PY{p}{)}\PY{o}{*}\PY{p}{(}\PY{l+m+mf}{1.}\PY{o}{+}\PY{n}{z2}\PY{p}{)}\PY{p}{,} \PY{n}{spec}\PY{o}{.}\PY{n}{spectrum}\PY{o}{/}\PY{n}{spec}\PY{o}{.}\PY{n}{spectrum}\PY{o}{.}\PY{n}{max}\PY{p}{(}\PY{p}{)}\PY{o}{/}\PY{l+m+mf}{1.5}\PY{p}{,} \PY{n}{color}\PY{o}{=}\PY{l+s+s1}{\PYZsq{}}\PY{l+s+s1}{red}\PY{l+s+s1}{\PYZsq{}}\PY{p}{)}
        \PY{n}{plt}\PY{o}{.}\PY{n}{title}\PY{p}{(}\PY{l+s+s1}{\PYZsq{}}\PY{l+s+s1}{Galaxy B}\PY{l+s+s1}{\PYZsq{}}\PY{p}{)}
        \PY{n}{xlabel}\PY{p}{(}\PY{l+s+s1}{\PYZsq{}}\PY{l+s+s1}{Wavelength}\PY{l+s+s1}{\PYZsq{}}\PY{p}{)}
        \PY{n}{ylabel}\PY{p}{(}\PY{l+s+s1}{\PYZsq{}}\PY{l+s+s1}{Brightness}\PY{l+s+s1}{\PYZsq{}}\PY{p}{)}
        \PY{n}{xlim}\PY{p}{(}\PY{l+m+mi}{3800}\PY{p}{,}\PY{l+m+mi}{6000}\PY{p}{)}
        \PY{n}{ylim}\PY{p}{(}\PY{o}{\PYZhy{}}\PY{l+m+mf}{0.1}\PY{p}{,} \PY{l+m+mf}{1.2}\PY{p}{)}
\end{Verbatim}


    \subsection{Questions:}\label{questions}

Which galaxy is closer to us?

Further away from us?

Now, let's check the redshift and learn more information about those two
galaxies. You need to COPY and PASTE the following:
http://cas.sdss.org/dr14/en/tools/explore/Summary.aspx in a new browser.
\textbf{Clicking on the link does not work.}\\
Click on "Search" on the left hand side menu bar, and then enter the
"Plate", "Fiber" and "MJD" for one galaxy at a time, and hit "Go". If
you click on the image, you can move around, zoom in and out - it's like
Google Maps for the night sky!

    \section{Redshift Ruler}\label{redshift-ruler}

Each group had a different mystery galaxy. Let's see how far away they
are compared with each other by placing them on the "redshift ruler" on
the white board at the front of the room (if applicable).

    \section{What's next?}\label{whats-next}

Well done! You have measured redshifts for two galaxies, which is how
astronomers determine distances to galaxies. Remember, the further away
a galaxy is from us, the faster is seems to be moving away from us, and
the more its light (spectrum) is redshifted! That's because the universe
is expanding.

Next, let's see what we can learn if we apply this information to many
galaxies. Onward to exploring our vast universe!

    \section{Activity 2: Look at the Position of Many
Galaxies}\label{activity-2-look-at-the-position-of-many-galaxies}

Similarly to using coordinates of latitude and longitude, the
coordinates on the sky are defined onto a sphere. They are called RA
(for Right Ascension) and Dec (for Declination). There are two
illustrations below of these coordinate systems.

Illustration of the celestial coordinate system with RA and Dec. You can
read here for an explanation by the SDSS team.

Illustration of the celestial coordinate system with RA and Dec. You can
read on Wikipedia about Right Ascension and Declination.

    \subsection{Step 2.1: Selecting Galaxies in a Region of the
Sky}\label{step-2.1-selecting-galaxies-in-a-region-of-the-sky}

Next, we will fetch the positions of galaxies on the sky, and plot their
RA and Dec coordinates.

    \begin{Verbatim}[commandchars=\\\{\}]
{\color{incolor}In [{\color{incolor} }]:} \PY{c+c1}{\PYZsh{}Code that we will need to fetch galaxies\PYZsq{} coordinates}
        \PY{k+kn}{from} \PY{n+nn}{astroML}\PY{n+nn}{.}\PY{n+nn}{datasets} \PY{k}{import} \PY{n}{fetch\PYZus{}sdss\PYZus{}specgals}
        \PY{k+kn}{import} \PY{n+nn}{matplotlib}\PY{n+nn}{.}\PY{n+nn}{cm} \PY{k}{as} \PY{n+nn}{cm}
        
        \PY{c+c1}{\PYZsh{}For 3D Plotting:}
        \PY{k+kn}{import} \PY{n+nn}{matplotlib}\PY{n+nn}{.}\PY{n+nn}{pyplot} \PY{k}{as} \PY{n+nn}{plt}
        \PY{k+kn}{from} \PY{n+nn}{mpl\PYZus{}toolkits}\PY{n+nn}{.}\PY{n+nn}{mplot3d} \PY{k}{import} \PY{n}{Axes3D}
\end{Verbatim}


    Now that the packages are loaded, run the cells below to actually fetch
the galaxy sample and plot their positions on the sky.

    \begin{Verbatim}[commandchars=\\\{\}]
{\color{incolor}In [{\color{incolor} }]:} \PY{c+c1}{\PYZsh{} Fetch the sample from the Sloan data}
        \PY{n}{data} \PY{o}{=} \PY{n}{fetch\PYZus{}sdss\PYZus{}specgals}\PY{p}{(}\PY{p}{)}
        \PY{n+nb}{print}\PY{p}{(}\PY{l+s+s1}{\PYZsq{}}\PY{l+s+s1}{Done retrieving the galaxy sample}\PY{l+s+s1}{\PYZsq{}}\PY{p}{)}
        
        \PY{c+c1}{\PYZsh{} Define the coordinate variables for plotting}
        \PY{n}{RA} \PY{o}{=} \PY{n}{data}\PY{p}{[}\PY{l+s+s1}{\PYZsq{}}\PY{l+s+s1}{ra}\PY{l+s+s1}{\PYZsq{}}\PY{p}{]}
        \PY{n}{DEC} \PY{o}{=} \PY{n}{data}\PY{p}{[}\PY{l+s+s1}{\PYZsq{}}\PY{l+s+s1}{dec}\PY{l+s+s1}{\PYZsq{}}\PY{p}{]}
        
        \PY{n+nb}{print}\PY{p}{(}\PY{l+s+s1}{\PYZsq{}}\PY{l+s+s1}{ }\PY{l+s+s1}{\PYZsq{}}\PY{p}{)}
        \PY{n+nb}{print}\PY{p}{(}\PY{l+s+s1}{\PYZsq{}}\PY{l+s+s1}{Range for RA values}\PY{l+s+s1}{\PYZsq{}}\PY{p}{)}
        \PY{n+nb}{print}\PY{p}{(}\PY{l+s+s1}{\PYZsq{}}\PY{l+s+s1}{  }\PY{l+s+s1}{\PYZsq{}}\PY{p}{,}\PY{n}{np}\PY{o}{.}\PY{n}{amin}\PY{p}{(}\PY{n}{RA}\PY{p}{)}\PY{p}{,}\PY{n}{np}\PY{o}{.}\PY{n}{amax}\PY{p}{(}\PY{n}{RA}\PY{p}{)}\PY{p}{)}
        \PY{n+nb}{print}\PY{p}{(}\PY{l+s+s1}{\PYZsq{}}\PY{l+s+s1}{Range for DEC values}\PY{l+s+s1}{\PYZsq{}}\PY{p}{)}
        \PY{n+nb}{print}\PY{p}{(}\PY{l+s+s1}{\PYZsq{}}\PY{l+s+s1}{  }\PY{l+s+s1}{\PYZsq{}}\PY{p}{,}\PY{n}{np}\PY{o}{.}\PY{n}{amin}\PY{p}{(}\PY{n}{DEC}\PY{p}{)}\PY{p}{,}\PY{n}{np}\PY{o}{.}\PY{n}{amax}\PY{p}{(}\PY{n}{DEC}\PY{p}{)}\PY{p}{)}
        \PY{n+nb}{print}\PY{p}{(}\PY{l+s+s1}{\PYZsq{}}\PY{l+s+s1}{ }\PY{l+s+s1}{\PYZsq{}}\PY{p}{)}
        
        \PY{c+c1}{\PYZsh{} convert RA range to [\PYZhy{}180,+180] instead of [0,360]}
        \PY{n}{RA} \PY{o}{\PYZhy{}}\PY{o}{=} \PY{l+m+mi}{180}
        
        \PY{n+nb}{print}\PY{p}{(}\PY{l+s+s1}{\PYZsq{}}\PY{l+s+s1}{Range for RA values after conversion}\PY{l+s+s1}{\PYZsq{}}\PY{p}{)}
        \PY{n+nb}{print}\PY{p}{(}\PY{l+s+s1}{\PYZsq{}}\PY{l+s+s1}{  }\PY{l+s+s1}{\PYZsq{}}\PY{p}{,}\PY{n}{np}\PY{o}{.}\PY{n}{amin}\PY{p}{(}\PY{n}{RA}\PY{p}{)}\PY{p}{,}\PY{n}{np}\PY{o}{.}\PY{n}{amax}\PY{p}{(}\PY{n}{RA}\PY{p}{)}\PY{p}{)}
        
        \PY{c+c1}{\PYZsh{}plot the RA/DEC positions}
        \PY{n}{figure}\PY{p}{(}\PY{p}{)}
        \PY{n}{s}\PY{o}{=}\PY{l+m+mf}{0.8}   \PY{c+c1}{\PYZsh{}symbol size}
        \PY{n}{plt}\PY{o}{.}\PY{n}{scatter}\PY{p}{(}\PY{n}{RA}\PY{p}{,} \PY{n}{DEC}\PY{p}{,}\PY{n}{s}\PY{o}{=}\PY{n}{s}\PY{p}{,} \PY{n}{color}\PY{o}{=}\PY{l+s+s1}{\PYZsq{}}\PY{l+s+s1}{red}\PY{l+s+s1}{\PYZsq{}}\PY{p}{,}\PY{n}{alpha}\PY{o}{=}\PY{l+m+mf}{0.3}\PY{p}{)}
        \PY{n}{plt}\PY{o}{.}\PY{n}{title}\PY{p}{(}\PY{l+s+s1}{\PYZsq{}}\PY{l+s+s1}{SDSS Galaxies}\PY{l+s+s1}{\PYZsq{}}\PY{p}{)}
        \PY{n}{xlabel}\PY{p}{(}\PY{l+s+s1}{\PYZsq{}}\PY{l+s+s1}{RA (degrees)}\PY{l+s+s1}{\PYZsq{}}\PY{p}{)}
        \PY{n}{ylabel}\PY{p}{(}\PY{l+s+s1}{\PYZsq{}}\PY{l+s+s1}{DEC (degrees)}\PY{l+s+s1}{\PYZsq{}}\PY{p}{)}
        
        \PY{c+c1}{\PYZsh{}range for the x axis (horizontal) and y axis (vertical)}
        \PY{n}{xlim}\PY{p}{(}\PY{l+m+mi}{0}\PY{p}{,}\PY{l+m+mi}{10}\PY{p}{)}
        \PY{n}{ylim}\PY{p}{(}\PY{l+m+mi}{10}\PY{p}{,}\PY{l+m+mi}{18}\PY{p}{)}
\end{Verbatim}


    \subsection{Step 2.2 Adding the 3rd
Dimension}\label{step-2.2-adding-the-3rd-dimension}

We saw before that in order to know the full distribution in 3D, we need
to know how far away the galaxies are located. Here, we will add the
information from the redshift. Remember: the larger the redshift, the
further away the galaxy!

First, we will plot all galaxies in red, and show galaxies that have
approximately the same redshift in black. In this first example, we will
select for values of redshift between z=0.10 and z=0.12. This is done by
computing galaxies in a window of -0.01 and +0.1 around z=0.11 (since
0.11-0.01=0.10 and 0.11+0.01=0.12). We call this an interval of
redshift.

    \begin{Verbatim}[commandchars=\\\{\}]
{\color{incolor}In [{\color{incolor} }]:} \PY{c+c1}{\PYZsh{}Fetch the sample from the Sloan data}
        \PY{n}{data} \PY{o}{=} \PY{n}{fetch\PYZus{}sdss\PYZus{}specgals}\PY{p}{(}\PY{p}{)}
        \PY{n+nb}{print}\PY{p}{(}\PY{l+s+s1}{\PYZsq{}}\PY{l+s+s1}{Done retrieving the galaxy sample}\PY{l+s+s1}{\PYZsq{}}\PY{p}{)}
        
        \PY{c+c1}{\PYZsh{}define the variables for plotting}
        \PY{n}{RA} \PY{o}{=} \PY{n}{data}\PY{p}{[}\PY{l+s+s1}{\PYZsq{}}\PY{l+s+s1}{ra}\PY{l+s+s1}{\PYZsq{}}\PY{p}{]}
        \PY{n}{DEC} \PY{o}{=} \PY{n}{data}\PY{p}{[}\PY{l+s+s1}{\PYZsq{}}\PY{l+s+s1}{dec}\PY{l+s+s1}{\PYZsq{}}\PY{p}{]}
        
        \PY{c+c1}{\PYZsh{} convert RA range to [\PYZhy{}180,+180] instead of [0,360]}
        \PY{n}{RA} \PY{o}{\PYZhy{}}\PY{o}{=} \PY{l+m+mi}{180}
        
        \PY{c+c1}{\PYZsh{}define redshift variable z}
        \PY{n}{z} \PY{o}{=} \PY{n}{data}\PY{p}{[}\PY{l+s+s1}{\PYZsq{}}\PY{l+s+s1}{z}\PY{l+s+s1}{\PYZsq{}}\PY{p}{]}
        
        \PY{c+c1}{\PYZsh{}pick a redshift range to highlight in a different color}
        \PY{n}{rz} \PY{o}{=} \PY{n}{np}\PY{o}{.}\PY{n}{where}\PY{p}{(}\PY{n}{np}\PY{o}{.}\PY{n}{absolute}\PY{p}{(}\PY{n}{z}\PY{o}{\PYZhy{}}\PY{l+m+mf}{0.08}\PY{p}{)}\PY{o}{\PYZlt{}}\PY{l+m+mf}{0.01}\PY{p}{)}
        
        \PY{c+c1}{\PYZsh{}plot the RA/DEC positions}
        \PY{n}{figure}\PY{p}{(}\PY{p}{)}
        \PY{n}{plt}\PY{o}{.}\PY{n}{scatter}\PY{p}{(}\PY{n}{RA}\PY{p}{,} \PY{n}{DEC}\PY{p}{,}\PY{n}{s}\PY{o}{=}\PY{l+m+mf}{5.0}\PY{p}{,}\PY{n}{lw}\PY{o}{=}\PY{l+m+mi}{0}\PY{p}{,}\PY{n}{c}\PY{o}{=}\PY{l+s+s1}{\PYZsq{}}\PY{l+s+s1}{red}\PY{l+s+s1}{\PYZsq{}}\PY{p}{,}\PY{n}{alpha}\PY{o}{=}\PY{l+m+mf}{0.3}\PY{p}{)}
        \PY{n}{plt}\PY{o}{.}\PY{n}{scatter}\PY{p}{(}\PY{n}{RA}\PY{p}{[}\PY{n}{rz}\PY{p}{]}\PY{p}{,} \PY{n}{DEC}\PY{p}{[}\PY{n}{rz}\PY{p}{]}\PY{p}{,}\PY{n}{s}\PY{o}{=}\PY{l+m+mf}{8.0}\PY{p}{,}\PY{n}{lw}\PY{o}{=}\PY{l+m+mi}{0}\PY{p}{,}\PY{n}{c}\PY{o}{=}\PY{l+s+s1}{\PYZsq{}}\PY{l+s+s1}{black}\PY{l+s+s1}{\PYZsq{}}\PY{p}{)}  \PY{c+c1}{\PYZsh{}rz selected galaxies}
        \PY{n}{plt}\PY{o}{.}\PY{n}{title}\PY{p}{(}\PY{l+s+s1}{\PYZsq{}}\PY{l+s+s1}{SDSS Galaxies (Thin Redshift Slice in Black)}\PY{l+s+s1}{\PYZsq{}}\PY{p}{)}
        \PY{n}{xlabel}\PY{p}{(}\PY{l+s+s1}{\PYZsq{}}\PY{l+s+s1}{RA (degrees)}\PY{l+s+s1}{\PYZsq{}}\PY{p}{)}
        \PY{n}{ylabel}\PY{p}{(}\PY{l+s+s1}{\PYZsq{}}\PY{l+s+s1}{DEC (degrees)}\PY{l+s+s1}{\PYZsq{}}\PY{p}{)}
        
        \PY{c+c1}{\PYZsh{}range for the x axis (horizontal) and y axis (vertical)}
        \PY{n}{xlim}\PY{p}{(}\PY{l+m+mi}{0}\PY{p}{,}\PY{l+m+mi}{10}\PY{p}{)}
        \PY{n}{ylim}\PY{p}{(}\PY{l+m+mi}{10}\PY{p}{,}\PY{l+m+mi}{18}\PY{p}{)}
\end{Verbatim}


    Now, instead of showing just one interval of redshift in black, we will
show the redshift of each galaxy color-coded. Each galaxy is shown with
a dot, and each dot will have a color corresponding to the redshift:
purple/blue colors mean a low redshift like between 0-0.05, then
green/yellow mean slightly higher redshift like 0.1, and so on until the
higher redshift shown here of 0.2 in red. Remember that this means that
points with exactly the same color are at the same distance from us!

    \begin{Verbatim}[commandchars=\\\{\}]
{\color{incolor}In [{\color{incolor} }]:} \PY{c+c1}{\PYZsh{}Fetch the sample from the Sloan data}
        \PY{n}{data} \PY{o}{=} \PY{n}{fetch\PYZus{}sdss\PYZus{}specgals}\PY{p}{(}\PY{p}{)}
        
        \PY{c+c1}{\PYZsh{}define the variables for plotting}
        \PY{n}{RA} \PY{o}{=} \PY{n}{data}\PY{p}{[}\PY{l+s+s1}{\PYZsq{}}\PY{l+s+s1}{ra}\PY{l+s+s1}{\PYZsq{}}\PY{p}{]}
        \PY{n}{DEC} \PY{o}{=} \PY{n}{data}\PY{p}{[}\PY{l+s+s1}{\PYZsq{}}\PY{l+s+s1}{dec}\PY{l+s+s1}{\PYZsq{}}\PY{p}{]}
        
        \PY{c+c1}{\PYZsh{} convert RA range to [\PYZhy{}180,+180] instead of [0,360]}
        \PY{n}{RA} \PY{o}{\PYZhy{}}\PY{o}{=} \PY{l+m+mi}{180}
        
        \PY{c+c1}{\PYZsh{}plot the RA/DEC positions}
        \PY{c+c1}{\PYZsh{}figure(figsize=(8,4.75))}
        \PY{n}{fig}\PY{p}{,} \PY{n}{ax} \PY{o}{=} \PY{n}{plt}\PY{o}{.}\PY{n}{subplots}\PY{p}{(}\PY{l+m+mi}{1}\PY{p}{,} \PY{l+m+mi}{1}\PY{p}{,} \PY{n}{figsize}\PY{o}{=}\PY{p}{(}\PY{l+m+mi}{8}\PY{p}{,} \PY{l+m+mf}{4.75}\PY{p}{)}\PY{p}{)}
        
        \PY{c+c1}{\PYZsh{} Tests to get black bckgrd}
        \PY{c+c1}{\PYZsh{}plt.style.use(\PYZsq{}default\PYZsq{})}
        \PY{c+c1}{\PYZsh{}plt.style.use(\PYZsq{}dark\PYZus{}background\PYZsq{})}
        \PY{c+c1}{\PYZsh{}plt.rcParams[\PYZsq{}figure.facecolor\PYZsq{}] = \PYZsq{}black\PYZsq{}}
        
        \PY{n}{s}\PY{o}{=}\PY{l+m+mf}{4.5}   \PY{c+c1}{\PYZsh{}symbol size}
        \PY{n}{plt}\PY{o}{.}\PY{n}{scatter}\PY{p}{(}\PY{n}{RA}\PY{p}{,} \PY{n}{DEC}\PY{p}{,}\PY{n}{s}\PY{o}{=}\PY{n}{s}\PY{p}{,}\PY{n}{c}\PY{o}{=}\PY{n}{data}\PY{p}{[}\PY{l+s+s1}{\PYZsq{}}\PY{l+s+s1}{z}\PY{l+s+s1}{\PYZsq{}}\PY{p}{]}\PY{p}{,} \PY{n}{lw}\PY{o}{=}\PY{l+m+mi}{0}\PY{p}{,}\PY{n}{cmap}\PY{o}{=}\PY{n}{plt}\PY{o}{.}\PY{n}{cm}\PY{o}{.}\PY{n}{rainbow}\PY{p}{,}
                    \PY{n}{vmin}\PY{o}{=}\PY{l+m+mi}{0}\PY{p}{,} \PY{n}{vmax}\PY{o}{=}\PY{l+m+mf}{0.2}\PY{p}{)}
        \PY{c+c1}{\PYZsh{} Add a rextangle to show where we will zoom in next:}
        \PY{c+c1}{\PYZsh{} We give the (x,y) of 4 corners + repeat first corner to close the rectangle}
        \PY{c+c1}{\PYZsh{}plt.plot([4,4,6,6,4], [11,13,13,11,11], color=\PYZsq{}black\PYZsq{}, linewidth=2)}
        
        \PY{c+c1}{\PYZsh{} Plot labels for the title and both horizontal (x) and vertical (y) axes}
        \PY{n}{plt}\PY{o}{.}\PY{n}{title}\PY{p}{(}\PY{l+s+s1}{\PYZsq{}}\PY{l+s+s1}{SDSS Galaxies (Color\PYZhy{}Coded by redshift)}\PY{l+s+s1}{\PYZsq{}}\PY{p}{)}
        \PY{n}{xlabel}\PY{p}{(}\PY{l+s+s1}{\PYZsq{}}\PY{l+s+s1}{RA (degrees)}\PY{l+s+s1}{\PYZsq{}}\PY{p}{)}
        \PY{n}{ylabel}\PY{p}{(}\PY{l+s+s1}{\PYZsq{}}\PY{l+s+s1}{DEC (degrees)}\PY{l+s+s1}{\PYZsq{}}\PY{p}{)}
        
        \PY{c+c1}{\PYZsh{}range for the x axis (horizontal) and y axis (vertical)}
        \PY{n}{xlim}\PY{p}{(}\PY{l+m+mi}{0}\PY{p}{,}\PY{l+m+mi}{10}\PY{p}{)}
        \PY{n}{ylim}\PY{p}{(}\PY{l+m+mi}{10}\PY{p}{,}\PY{l+m+mi}{18}\PY{p}{)}
        
        \PY{c+c1}{\PYZsh{}color bar}
        \PY{n}{plt}\PY{o}{.}\PY{n}{colorbar}\PY{p}{(}\PY{p}{)}
        
        \PY{n}{plt}\PY{o}{.}\PY{n}{show}\PY{p}{(}\PY{p}{)}
\end{Verbatim}


    \subsubsection{Next, you will make a 3D version of the above 2D plot
(same dots but in 3D). You may need to execute the cell
twice.}\label{next-you-will-make-a-3d-version-of-the-above-2d-plot-same-dots-but-in-3d.-you-may-need-to-execute-the-cell-twice.}

    \begin{Verbatim}[commandchars=\\\{\}]
{\color{incolor}In [{\color{incolor} }]:} \PY{c+c1}{\PYZsh{}3D Plotting}
        
        \PY{c+c1}{\PYZsh{}Fetch the sample from the Sloan data}
        \PY{n}{data} \PY{o}{=} \PY{n}{fetch\PYZus{}sdss\PYZus{}specgals}\PY{p}{(}\PY{p}{)}
        
        \PY{c+c1}{\PYZsh{}define the variables for plotting}
        \PY{n}{RA} \PY{o}{=} \PY{n}{data}\PY{p}{[}\PY{l+s+s1}{\PYZsq{}}\PY{l+s+s1}{ra}\PY{l+s+s1}{\PYZsq{}}\PY{p}{]}
        \PY{n}{RA} \PY{o}{\PYZhy{}}\PY{o}{=} \PY{l+m+mi}{180} \PY{c+c1}{\PYZsh{} convert RA range to [\PYZhy{}180,+180] instead of [0,360]}
        \PY{n}{DEC} \PY{o}{=} \PY{n}{data}\PY{p}{[}\PY{l+s+s1}{\PYZsq{}}\PY{l+s+s1}{dec}\PY{l+s+s1}{\PYZsq{}}\PY{p}{]}
        \PY{n}{z} \PY{o}{=} \PY{n}{data}\PY{p}{[}\PY{l+s+s1}{\PYZsq{}}\PY{l+s+s1}{z}\PY{l+s+s1}{\PYZsq{}}\PY{p}{]}
        
        \PY{c+c1}{\PYZsh{} Setting ranges}
        \PY{n}{limits} \PY{o}{=} \PY{n}{np}\PY{o}{.}\PY{n}{logical\PYZus{}and}\PY{p}{(}\PY{n}{np}\PY{o}{.}\PY{n}{logical\PYZus{}and}\PY{p}{(}\PY{n}{np}\PY{o}{.}\PY{n}{logical\PYZus{}and}\PY{p}{(}\PY{n}{RA}\PY{o}{\PYZlt{}}\PY{l+m+mi}{10}\PY{p}{,} \PY{n}{RA}\PY{o}{\PYZgt{}}\PY{l+m+mi}{0}\PY{p}{)}\PY{p}{,} \PY{n}{np}\PY{o}{.}\PY{n}{logical\PYZus{}and}\PY{p}{(}\PY{n}{DEC}\PY{o}{\PYZgt{}}\PY{l+m+mi}{10}\PY{p}{,} \PY{n}{DEC}\PY{o}{\PYZlt{}}\PY{l+m+mi}{18}\PY{p}{)}\PY{p}{)}\PY{p}{,} \PY{n}{np}\PY{o}{.}\PY{n}{logical\PYZus{}and}\PY{p}{(}\PY{n}{z}\PY{o}{\PYZgt{}}\PY{l+m+mf}{0.0}\PY{p}{,} \PY{n}{z}\PY{o}{\PYZlt{}}\PY{l+m+mf}{0.2}\PY{p}{)}\PY{p}{)}
        
        \PY{n}{RA\PYZus{}new} \PY{o}{=} \PY{n}{RA}\PY{p}{[}\PY{n}{limits}\PY{p}{]}
        \PY{n}{DEC\PYZus{}new} \PY{o}{=} \PY{n}{DEC}\PY{p}{[}\PY{n}{limits}\PY{p}{]}
        
        \PY{c+c1}{\PYZsh{}define redshift variable z}
        
        \PY{n}{z\PYZus{}new} \PY{o}{=} \PY{n}{z}\PY{p}{[}\PY{n}{limits}\PY{p}{]}
        
        \PY{c+c1}{\PYZsh{}plot the RA/DEC positions}
        \PY{n}{fig} \PY{o}{=} \PY{n}{plt}\PY{o}{.}\PY{n}{figure}\PY{p}{(}\PY{n}{figsize}\PY{o}{=}\PY{p}{(}\PY{l+m+mi}{8}\PY{p}{,}\PY{l+m+mi}{5}\PY{p}{)}\PY{p}{)}
        \PY{n}{ax} \PY{o}{=} \PY{n}{fig}\PY{o}{.}\PY{n}{add\PYZus{}subplot}\PY{p}{(}\PY{l+m+mi}{111}\PY{p}{,} \PY{n}{projection}\PY{o}{=}\PY{l+s+s1}{\PYZsq{}}\PY{l+s+s1}{3d}\PY{l+s+s1}{\PYZsq{}}\PY{p}{)}
        \PY{n}{p} \PY{o}{=} \PY{n}{ax}\PY{o}{.}\PY{n}{scatter}\PY{p}{(}\PY{n}{RA\PYZus{}new}\PY{p}{,} \PY{n}{z\PYZus{}new}\PY{p}{,} \PY{n}{DEC\PYZus{}new}\PY{p}{,} \PY{n}{zdir}\PY{o}{=}\PY{l+s+s1}{\PYZsq{}}\PY{l+s+s1}{z}\PY{l+s+s1}{\PYZsq{}}\PY{p}{,} \PY{n}{s}\PY{o}{=}\PY{l+m+mf}{2.5}\PY{p}{,} \PY{n}{c}\PY{o}{=}\PY{n}{z\PYZus{}new}\PY{p}{,} \PY{n}{cmap}\PY{o}{=}\PY{n}{plt}\PY{o}{.}\PY{n}{cm}\PY{o}{.}\PY{n}{rainbow}\PY{p}{,} \PY{n}{vmin}\PY{o}{=}\PY{l+m+mi}{0}\PY{p}{,} \PY{n}{vmax}\PY{o}{=}\PY{l+m+mf}{0.2}\PY{p}{,} \PY{n}{depthshade}\PY{o}{=}\PY{k+kc}{True}\PY{p}{)}
        \PY{n}{fig}\PY{o}{.}\PY{n}{colorbar}\PY{p}{(}\PY{n}{p}\PY{p}{)}
        
        \PY{c+c1}{\PYZsh{} Plot labels for the title and both horizontal (x) and vertical (y) axes}
        \PY{n}{plt}\PY{o}{.}\PY{n}{title}\PY{p}{(}\PY{l+s+s1}{\PYZsq{}}\PY{l+s+s1}{SDSS Galaxies (Color\PYZhy{}Coded by redshift)}\PY{l+s+s1}{\PYZsq{}}\PY{p}{)}
        \PY{n}{ax}\PY{o}{.}\PY{n}{set\PYZus{}xlabel}\PY{p}{(}\PY{l+s+s1}{\PYZsq{}}\PY{l+s+s1}{RA (deg)}\PY{l+s+s1}{\PYZsq{}}\PY{p}{)}
        \PY{n}{ax}\PY{o}{.}\PY{n}{set\PYZus{}ylabel}\PY{p}{(}\PY{l+s+s1}{\PYZsq{}}\PY{l+s+s1}{Redshift (z)}\PY{l+s+s1}{\PYZsq{}}\PY{p}{)}
        \PY{n}{ax}\PY{o}{.}\PY{n}{set\PYZus{}zlabel}\PY{p}{(}\PY{l+s+s1}{\PYZsq{}}\PY{l+s+s1}{DEC (deg)}\PY{l+s+s1}{\PYZsq{}}\PY{p}{)}
        
        \PY{c+c1}{\PYZsh{}range for the x axis (horizontal) and y axis (vertical)}
        \PY{c+c1}{\PYZsh{} ax.set\PYZus{}xlim3d(0,10)}
        \PY{c+c1}{\PYZsh{} ax.set\PYZus{}zlim3d(10,18)}
        
        \PY{n}{plt}\PY{o}{.}\PY{n}{show}\PY{p}{(}\PY{p}{)}
\end{Verbatim}


    The color bar to the right-hand side shows the correspondence between
color and redshift. As mentioned before, points with exactly the same
color are at the same distance from us. Purple points are the closest to
us, then blue, aqua, green and so on. Think about which galaxies/colors
are near and which galaxies/colors are far.

\subsection{Questions:}\label{questions}

Can you use this information to imagine the distribution of galaxies in
3D?

Do you notice any structure together at the same distance from us?

The black rectangle in the figure shows where we will zoom in during the
next exercise below.

    \subsection{Step 2.3 Zooming In and Zooming
Out}\label{step-2.3-zooming-in-and-zooming-out}

Now, we will repeat the plots from Step 2.2 above, but with a zoom on a
smaller region ("zooming in"), and then over a larger region ("zooming
out").

    \begin{Verbatim}[commandchars=\\\{\}]
{\color{incolor}In [{\color{incolor} }]:} \PY{c+c1}{\PYZsh{} ZOOMING IN}
        
        \PY{c+c1}{\PYZsh{}Fetch the sample from the Sloan data}
        \PY{n}{data} \PY{o}{=} \PY{n}{fetch\PYZus{}sdss\PYZus{}specgals}\PY{p}{(}\PY{p}{)}
        
        \PY{c+c1}{\PYZsh{}define the variables for plotting}
        \PY{n}{RA} \PY{o}{=} \PY{n}{data}\PY{p}{[}\PY{l+s+s1}{\PYZsq{}}\PY{l+s+s1}{ra}\PY{l+s+s1}{\PYZsq{}}\PY{p}{]}
        \PY{n}{DEC} \PY{o}{=} \PY{n}{data}\PY{p}{[}\PY{l+s+s1}{\PYZsq{}}\PY{l+s+s1}{dec}\PY{l+s+s1}{\PYZsq{}}\PY{p}{]}
        
        \PY{c+c1}{\PYZsh{} convert RA range to [\PYZhy{}180,+180] instead of [0,360]}
        \PY{n}{RA} \PY{o}{\PYZhy{}}\PY{o}{=} \PY{l+m+mi}{180}
        
        \PY{c+c1}{\PYZsh{}plot the RA/DEC positions}
        \PY{n}{figure}\PY{p}{(}\PY{p}{)}
        \PY{n}{s}\PY{o}{=}\PY{l+m+mf}{10.0}   \PY{c+c1}{\PYZsh{}symbol size}
        \PY{n}{plt}\PY{o}{.}\PY{n}{scatter}\PY{p}{(}\PY{n}{RA}\PY{p}{,} \PY{n}{DEC}\PY{p}{,}\PY{n}{s}\PY{o}{=}\PY{n}{s}\PY{p}{,}\PY{n}{c}\PY{o}{=}\PY{n}{data}\PY{p}{[}\PY{l+s+s1}{\PYZsq{}}\PY{l+s+s1}{z}\PY{l+s+s1}{\PYZsq{}}\PY{p}{]}\PY{p}{,} \PY{n}{lw}\PY{o}{=}\PY{l+m+mi}{0}\PY{p}{,}\PY{n}{cmap}\PY{o}{=}\PY{n}{plt}\PY{o}{.}\PY{n}{cm}\PY{o}{.}\PY{n}{rainbow}\PY{p}{,}
                    \PY{n}{vmin}\PY{o}{=}\PY{l+m+mi}{0}\PY{p}{,} \PY{n}{vmax}\PY{o}{=}\PY{l+m+mf}{0.2}\PY{p}{)}
        \PY{n}{plt}\PY{o}{.}\PY{n}{title}\PY{p}{(}\PY{l+s+s1}{\PYZsq{}}\PY{l+s+s1}{SDSS Galaxies (Zoom In)}\PY{l+s+s1}{\PYZsq{}}\PY{p}{)}
        \PY{n}{xlabel}\PY{p}{(}\PY{l+s+s1}{\PYZsq{}}\PY{l+s+s1}{RA (degrees)}\PY{l+s+s1}{\PYZsq{}}\PY{p}{)}
        \PY{n}{ylabel}\PY{p}{(}\PY{l+s+s1}{\PYZsq{}}\PY{l+s+s1}{DEC (degrees)}\PY{l+s+s1}{\PYZsq{}}\PY{p}{)}
        
        \PY{c+c1}{\PYZsh{}range for the x axis (horizontal) and y axis (vertical)}
        \PY{n}{xlim}\PY{p}{(}\PY{l+m+mi}{4}\PY{p}{,}\PY{l+m+mi}{6}\PY{p}{)}
        \PY{n}{ylim}\PY{p}{(}\PY{l+m+mi}{11}\PY{p}{,}\PY{l+m+mi}{13}\PY{p}{)}
        
        \PY{c+c1}{\PYZsh{}color bar}
        \PY{n}{plt}\PY{o}{.}\PY{n}{colorbar}\PY{p}{(}\PY{p}{)}
        
        \PY{c+c1}{\PYZsh{} Count how many galaxies are within the plot}
        \PY{n}{points} \PY{o}{=} \PY{n}{np}\PY{o}{.}\PY{n}{column\PYZus{}stack}\PY{p}{(}\PY{p}{[}\PY{n}{RA}\PY{p}{,} \PY{n}{DEC}\PY{p}{]}\PY{p}{)}
        \PY{n}{verts} \PY{o}{=} \PY{n}{np}\PY{o}{.}\PY{n}{array}\PY{p}{(}\PY{p}{[}\PY{p}{[}\PY{l+m+mi}{4}\PY{p}{,}\PY{l+m+mi}{4}\PY{p}{,}\PY{l+m+mi}{6}\PY{p}{,}\PY{l+m+mi}{6}\PY{p}{]}\PY{p}{,} \PY{p}{[}\PY{l+m+mi}{11}\PY{p}{,}\PY{l+m+mi}{13}\PY{p}{,}\PY{l+m+mi}{13}\PY{p}{,}\PY{l+m+mi}{11}\PY{p}{]}\PY{p}{]}\PY{p}{)}\PY{o}{.}\PY{n}{T}
        \PY{n}{path} \PY{o}{=} \PY{n}{mpath}\PY{o}{.}\PY{n}{Path}\PY{p}{(}\PY{n}{verts}\PY{p}{)}
        \PY{n}{points\PYZus{}inside} \PY{o}{=} \PY{n}{points}\PY{p}{[}\PY{n}{path}\PY{o}{.}\PY{n}{contains\PYZus{}points}\PY{p}{(}\PY{n}{points}\PY{p}{)}\PY{p}{]}
        \PY{n+nb}{print}\PY{p}{(}\PY{l+s+s1}{\PYZsq{}}\PY{l+s+s1}{\PYZhy{}\PYZhy{}\PYZhy{}\PYZhy{}\PYZhy{}\PYZhy{}\PYZhy{}\PYZhy{}\PYZhy{}\PYZhy{}\PYZhy{}\PYZhy{}\PYZhy{}\PYZhy{}\PYZhy{}\PYZhy{}\PYZhy{}\PYZhy{}\PYZhy{}\PYZhy{}\PYZhy{}\PYZhy{}\PYZhy{}\PYZhy{}\PYZhy{}\PYZhy{}\PYZhy{}\PYZhy{}\PYZhy{}\PYZhy{}\PYZhy{}\PYZhy{}\PYZhy{}\PYZhy{}\PYZhy{}\PYZhy{}\PYZhy{}}\PY{l+s+s1}{\PYZsq{}}\PY{p}{)}
        \PY{n+nb}{print}\PY{p}{(}\PY{l+s+s1}{\PYZsq{}}\PY{l+s+s1}{Number of galaxies in the plot above:}\PY{l+s+s1}{\PYZsq{}}\PY{p}{)}
        \PY{n+nb}{print}\PY{p}{(}\PY{l+s+s1}{\PYZsq{}}\PY{l+s+s1}{  }\PY{l+s+s1}{\PYZsq{}}\PY{p}{,}\PY{n}{np}\PY{o}{.}\PY{n}{count\PYZus{}nonzero}\PY{p}{(}\PY{n}{points\PYZus{}inside}\PY{p}{)}\PY{p}{)}
\end{Verbatim}


    \subsubsection{Next, you will make a 3D version of the above 2D plot
(same dots but in 3D). You may need to execute the cell
twice.}\label{next-you-will-make-a-3d-version-of-the-above-2d-plot-same-dots-but-in-3d.-you-may-need-to-execute-the-cell-twice.}

    \begin{Verbatim}[commandchars=\\\{\}]
{\color{incolor}In [{\color{incolor} }]:} \PY{c+c1}{\PYZsh{}3D Plotting}
        
        \PY{c+c1}{\PYZsh{}Fetch the sample from the Sloan data}
        \PY{n}{data} \PY{o}{=} \PY{n}{fetch\PYZus{}sdss\PYZus{}specgals}\PY{p}{(}\PY{p}{)}
        
        \PY{c+c1}{\PYZsh{}define the variables for plotting}
        \PY{n}{RA} \PY{o}{=} \PY{n}{data}\PY{p}{[}\PY{l+s+s1}{\PYZsq{}}\PY{l+s+s1}{ra}\PY{l+s+s1}{\PYZsq{}}\PY{p}{]}
        \PY{n}{RA} \PY{o}{\PYZhy{}}\PY{o}{=} \PY{l+m+mi}{180} \PY{c+c1}{\PYZsh{} convert RA range to [\PYZhy{}180,+180] instead of [0,360]}
        \PY{n}{DEC} \PY{o}{=} \PY{n}{data}\PY{p}{[}\PY{l+s+s1}{\PYZsq{}}\PY{l+s+s1}{dec}\PY{l+s+s1}{\PYZsq{}}\PY{p}{]}
        \PY{n}{z} \PY{o}{=} \PY{n}{data}\PY{p}{[}\PY{l+s+s1}{\PYZsq{}}\PY{l+s+s1}{z}\PY{l+s+s1}{\PYZsq{}}\PY{p}{]}
        
        \PY{c+c1}{\PYZsh{} Setting ranges}
        \PY{n}{limits} \PY{o}{=} \PY{n}{np}\PY{o}{.}\PY{n}{logical\PYZus{}and}\PY{p}{(}\PY{n}{np}\PY{o}{.}\PY{n}{logical\PYZus{}and}\PY{p}{(}\PY{n}{np}\PY{o}{.}\PY{n}{logical\PYZus{}and}\PY{p}{(}\PY{n}{RA}\PY{o}{\PYZlt{}}\PY{l+m+mi}{6}\PY{p}{,} \PY{n}{RA}\PY{o}{\PYZgt{}}\PY{l+m+mi}{4}\PY{p}{)}\PY{p}{,} \PY{n}{np}\PY{o}{.}\PY{n}{logical\PYZus{}and}\PY{p}{(}\PY{n}{DEC}\PY{o}{\PYZgt{}}\PY{l+m+mi}{11}\PY{p}{,} \PY{n}{DEC}\PY{o}{\PYZlt{}}\PY{l+m+mi}{13}\PY{p}{)}\PY{p}{)}\PY{p}{,} \PY{n}{np}\PY{o}{.}\PY{n}{logical\PYZus{}and}\PY{p}{(}\PY{n}{z}\PY{o}{\PYZgt{}}\PY{l+m+mf}{0.0}\PY{p}{,} \PY{n}{z}\PY{o}{\PYZlt{}}\PY{l+m+mf}{0.2}\PY{p}{)}\PY{p}{)}
        
        \PY{n}{RA\PYZus{}new} \PY{o}{=} \PY{n}{RA}\PY{p}{[}\PY{n}{limits}\PY{p}{]}
        \PY{n}{DEC\PYZus{}new} \PY{o}{=} \PY{n}{DEC}\PY{p}{[}\PY{n}{limits}\PY{p}{]}
        
        \PY{c+c1}{\PYZsh{}define redshift variable z}
        
        \PY{n}{z\PYZus{}new} \PY{o}{=} \PY{n}{z}\PY{p}{[}\PY{n}{limits}\PY{p}{]}
        
        \PY{c+c1}{\PYZsh{}plot the RA/DEC positions}
        \PY{n}{fig} \PY{o}{=} \PY{n}{plt}\PY{o}{.}\PY{n}{figure}\PY{p}{(}\PY{n}{figsize}\PY{o}{=}\PY{p}{(}\PY{l+m+mi}{8}\PY{p}{,}\PY{l+m+mf}{4.75}\PY{p}{)}\PY{p}{)}
        \PY{n}{ax} \PY{o}{=} \PY{n}{fig}\PY{o}{.}\PY{n}{add\PYZus{}subplot}\PY{p}{(}\PY{l+m+mi}{111}\PY{p}{,} \PY{n}{projection}\PY{o}{=}\PY{l+s+s1}{\PYZsq{}}\PY{l+s+s1}{3d}\PY{l+s+s1}{\PYZsq{}}\PY{p}{)}
        \PY{n}{p} \PY{o}{=} \PY{n}{ax}\PY{o}{.}\PY{n}{scatter}\PY{p}{(}\PY{n}{RA\PYZus{}new}\PY{p}{,} \PY{n}{z\PYZus{}new}\PY{p}{,} \PY{n}{DEC\PYZus{}new}\PY{p}{,} \PY{n}{zdir}\PY{o}{=}\PY{l+s+s1}{\PYZsq{}}\PY{l+s+s1}{z}\PY{l+s+s1}{\PYZsq{}}\PY{p}{,} \PY{n}{s}\PY{o}{=}\PY{l+m+mi}{10}\PY{p}{,} \PY{n}{c}\PY{o}{=}\PY{n}{z\PYZus{}new}\PY{p}{,} \PY{n}{cmap}\PY{o}{=}\PY{n}{plt}\PY{o}{.}\PY{n}{cm}\PY{o}{.}\PY{n}{rainbow}\PY{p}{,} \PY{n}{vmin}\PY{o}{=}\PY{l+m+mi}{0}\PY{p}{,} \PY{n}{vmax}\PY{o}{=}\PY{l+m+mf}{0.2}\PY{p}{,} \PY{n}{depthshade}\PY{o}{=}\PY{k+kc}{True}\PY{p}{)}
        \PY{n}{fig}\PY{o}{.}\PY{n}{colorbar}\PY{p}{(}\PY{n}{p}\PY{p}{)}
        
        \PY{c+c1}{\PYZsh{} Plot labels for the title and both horizontal (x) and vertical (y) axes}
        \PY{n}{plt}\PY{o}{.}\PY{n}{title}\PY{p}{(}\PY{l+s+s1}{\PYZsq{}}\PY{l+s+s1}{SDSS Galaxies (Color\PYZhy{}Coded by redshift)}\PY{l+s+s1}{\PYZsq{}}\PY{p}{)}
        \PY{n}{ax}\PY{o}{.}\PY{n}{set\PYZus{}xlabel}\PY{p}{(}\PY{l+s+s1}{\PYZsq{}}\PY{l+s+s1}{RA (deg)}\PY{l+s+s1}{\PYZsq{}}\PY{p}{)}
        \PY{n}{ax}\PY{o}{.}\PY{n}{set\PYZus{}ylabel}\PY{p}{(}\PY{l+s+s1}{\PYZsq{}}\PY{l+s+s1}{Redshift (z)}\PY{l+s+s1}{\PYZsq{}}\PY{p}{)}
        \PY{n}{ax}\PY{o}{.}\PY{n}{set\PYZus{}zlabel}\PY{p}{(}\PY{l+s+s1}{\PYZsq{}}\PY{l+s+s1}{DEC (deg)}\PY{l+s+s1}{\PYZsq{}}\PY{p}{)}
        
        \PY{n}{plt}\PY{o}{.}\PY{n}{show}\PY{p}{(}\PY{p}{)}
\end{Verbatim}


    \subsection{Questions:}\label{questions}

What do see?

Any interesting galaxy structures?

What galaxy structures are closer/further from you?

 \#\#\# Now, let's step back and plot galaxies over a large region of
the sky!

    \begin{Verbatim}[commandchars=\\\{\}]
{\color{incolor}In [{\color{incolor} }]:} \PY{c+c1}{\PYZsh{} ZOOMING OUT}
        
        \PY{c+c1}{\PYZsh{}Fetch the sample from the Sloan data}
        \PY{n}{data} \PY{o}{=} \PY{n}{fetch\PYZus{}sdss\PYZus{}specgals}\PY{p}{(}\PY{p}{)}
        
        \PY{c+c1}{\PYZsh{}define the variables for plotting}
        \PY{n}{RA} \PY{o}{=} \PY{n}{data}\PY{p}{[}\PY{l+s+s1}{\PYZsq{}}\PY{l+s+s1}{ra}\PY{l+s+s1}{\PYZsq{}}\PY{p}{]}
        \PY{n}{DEC} \PY{o}{=} \PY{n}{data}\PY{p}{[}\PY{l+s+s1}{\PYZsq{}}\PY{l+s+s1}{dec}\PY{l+s+s1}{\PYZsq{}}\PY{p}{]}
        
        \PY{c+c1}{\PYZsh{} convert RA range to [\PYZhy{}180,+180] instead of [0,360]}
        \PY{n}{RA} \PY{o}{\PYZhy{}}\PY{o}{=} \PY{l+m+mi}{180}
        
        \PY{c+c1}{\PYZsh{}plot the RA/DEC positions}
        \PY{n}{figure}\PY{p}{(}\PY{p}{)}
        \PY{n}{s}\PY{o}{=}\PY{l+m+mf}{1.0}   \PY{c+c1}{\PYZsh{}symbol size}
        \PY{n}{plt}\PY{o}{.}\PY{n}{scatter}\PY{p}{(}\PY{n}{RA}\PY{p}{,} \PY{n}{DEC}\PY{p}{,}\PY{n}{s}\PY{o}{=}\PY{n}{s}\PY{p}{,}\PY{n}{c}\PY{o}{=}\PY{n}{data}\PY{p}{[}\PY{l+s+s1}{\PYZsq{}}\PY{l+s+s1}{z}\PY{l+s+s1}{\PYZsq{}}\PY{p}{]}\PY{p}{,} \PY{n}{lw}\PY{o}{=}\PY{l+m+mi}{0}\PY{p}{,}\PY{n}{cmap}\PY{o}{=}\PY{n}{plt}\PY{o}{.}\PY{n}{cm}\PY{o}{.}\PY{n}{rainbow}\PY{p}{,}
                    \PY{n}{vmin}\PY{o}{=}\PY{l+m+mi}{0}\PY{p}{,} \PY{n}{vmax}\PY{o}{=}\PY{l+m+mf}{0.2}\PY{p}{)}
        
        \PY{c+c1}{\PYZsh{} Add a rextangle where we zoomed in before:}
        \PY{c+c1}{\PYZsh{} We give the (x,y) of 4 corners + repeat first corner to close the rectangle}
        \PY{c+c1}{\PYZsh{}plt.plot([4,4,6,6,4], [11,13,13,11,11], color=\PYZsq{}black\PYZsq{}, linewidth=1)}
        
        \PY{c+c1}{\PYZsh{} Add original rectangle (before zooming in)}
        \PY{n}{plt}\PY{o}{.}\PY{n}{plot}\PY{p}{(}\PY{p}{[}\PY{l+m+mi}{0}\PY{p}{,}\PY{l+m+mi}{0}\PY{p}{,}\PY{l+m+mi}{10}\PY{p}{,}\PY{l+m+mi}{10}\PY{p}{,}\PY{l+m+mi}{0}\PY{p}{]}\PY{p}{,} \PY{p}{[}\PY{l+m+mi}{10}\PY{p}{,}\PY{l+m+mi}{18}\PY{p}{,}\PY{l+m+mi}{18}\PY{p}{,}\PY{l+m+mi}{10}\PY{p}{,}\PY{l+m+mi}{10}\PY{p}{]}\PY{p}{,} \PY{n}{color}\PY{o}{=}\PY{l+s+s1}{\PYZsq{}}\PY{l+s+s1}{black}\PY{l+s+s1}{\PYZsq{}}\PY{p}{,} \PY{n}{linewidth}\PY{o}{=}\PY{l+m+mi}{2}\PY{p}{)}
        
        
        \PY{c+c1}{\PYZsh{} Plot labels for the title and both horizontal (x) and vertical (y) axes}
        \PY{n}{plt}\PY{o}{.}\PY{n}{title}\PY{p}{(}\PY{l+s+s1}{\PYZsq{}}\PY{l+s+s1}{SDSS Galaxies (Zoom Out)}\PY{l+s+s1}{\PYZsq{}}\PY{p}{)}
        \PY{n}{xlabel}\PY{p}{(}\PY{l+s+s1}{\PYZsq{}}\PY{l+s+s1}{RA (degrees)}\PY{l+s+s1}{\PYZsq{}}\PY{p}{)}
        \PY{n}{ylabel}\PY{p}{(}\PY{l+s+s1}{\PYZsq{}}\PY{l+s+s1}{DEC (degrees)}\PY{l+s+s1}{\PYZsq{}}\PY{p}{)}
        
        \PY{c+c1}{\PYZsh{}range for the x axis (horizontal) and y axis (vertical)}
        \PY{n}{xlim}\PY{p}{(}\PY{o}{\PYZhy{}}\PY{l+m+mi}{15}\PY{p}{,}\PY{l+m+mi}{15}\PY{p}{)}
        \PY{n}{ylim}\PY{p}{(}\PY{l+m+mi}{0}\PY{p}{,}\PY{l+m+mi}{30}\PY{p}{)}
        
        \PY{c+c1}{\PYZsh{}color bar}
        \PY{n}{plt}\PY{o}{.}\PY{n}{colorbar}\PY{p}{(}\PY{p}{)}
        
        \PY{c+c1}{\PYZsh{} Count how many galaxies are within the plot}
        \PY{n}{points} \PY{o}{=} \PY{n}{np}\PY{o}{.}\PY{n}{column\PYZus{}stack}\PY{p}{(}\PY{p}{[}\PY{n}{RA}\PY{p}{,} \PY{n}{DEC}\PY{p}{]}\PY{p}{)}
        \PY{n}{verts} \PY{o}{=} \PY{n}{np}\PY{o}{.}\PY{n}{array}\PY{p}{(}\PY{p}{[}\PY{p}{[}\PY{o}{\PYZhy{}}\PY{l+m+mi}{15}\PY{p}{,}\PY{o}{\PYZhy{}}\PY{l+m+mi}{15}\PY{p}{,}\PY{l+m+mi}{15}\PY{p}{,}\PY{l+m+mi}{15}\PY{p}{]}\PY{p}{,} \PY{p}{[}\PY{l+m+mi}{0}\PY{p}{,}\PY{l+m+mi}{30}\PY{p}{,}\PY{l+m+mi}{30}\PY{p}{,}\PY{l+m+mi}{0}\PY{p}{]}\PY{p}{]}\PY{p}{)}\PY{o}{.}\PY{n}{T}
        \PY{n}{path} \PY{o}{=} \PY{n}{mpath}\PY{o}{.}\PY{n}{Path}\PY{p}{(}\PY{n}{verts}\PY{p}{)}
        \PY{n}{points\PYZus{}inside} \PY{o}{=} \PY{n}{points}\PY{p}{[}\PY{n}{path}\PY{o}{.}\PY{n}{contains\PYZus{}points}\PY{p}{(}\PY{n}{points}\PY{p}{)}\PY{p}{]}
        \PY{n+nb}{print}\PY{p}{(}\PY{l+s+s1}{\PYZsq{}}\PY{l+s+s1}{\PYZhy{}\PYZhy{}\PYZhy{}\PYZhy{}\PYZhy{}\PYZhy{}\PYZhy{}\PYZhy{}\PYZhy{}\PYZhy{}\PYZhy{}\PYZhy{}\PYZhy{}\PYZhy{}\PYZhy{}\PYZhy{}\PYZhy{}\PYZhy{}\PYZhy{}\PYZhy{}\PYZhy{}\PYZhy{}\PYZhy{}\PYZhy{}\PYZhy{}\PYZhy{}\PYZhy{}\PYZhy{}\PYZhy{}\PYZhy{}\PYZhy{}\PYZhy{}\PYZhy{}\PYZhy{}\PYZhy{}\PYZhy{}\PYZhy{}}\PY{l+s+s1}{\PYZsq{}}\PY{p}{)}
        \PY{n+nb}{print}\PY{p}{(}\PY{l+s+s1}{\PYZsq{}}\PY{l+s+s1}{Number of galaxies in the plot above:}\PY{l+s+s1}{\PYZsq{}}\PY{p}{)}
        \PY{n+nb}{print}\PY{p}{(}\PY{l+s+s1}{\PYZsq{}}\PY{l+s+s1}{  }\PY{l+s+s1}{\PYZsq{}}\PY{p}{,}\PY{n}{np}\PY{o}{.}\PY{n}{count\PYZus{}nonzero}\PY{p}{(}\PY{n}{points\PYZus{}inside}\PY{p}{)}\PY{p}{)}
\end{Verbatim}


    The color bar to the right-hand side shows the correspondence between
color and redshift. The black rectangle shows the region from the
previous plot ("Zoomed In"). You can compare the size of the two regions
directly.

\subsection{Questions:}\label{questions}

How many times more galaxies are in the large (zoomed out) view relative
to the small (zoomed in) view?

How many times can you fit the small region within the large region?
(Hint: compute the size from the axes)

Are those two numbers above the same? What does it mean?

What do you see now on the zoomed out view?

Are those structures smaller or larger?

    \subsection{Bonus: Plot Full Sample over Sky
Projection}\label{bonus-plot-full-sample-over-sky-projection}

Below, we will again plot the positions of galaxies, and include the
information on redshift as the color (but with a different color
scheme).

The difference with the steps above is that we will now plot the sample
of galaxies over the full sky. The SDSS survey does not cover the full
sky, so we will see what we call the "footprint" of the survey. This
means the regions of the sky where the telescope was pointed to gather
images and spectra.

    \begin{Verbatim}[commandchars=\\\{\}]
{\color{incolor}In [{\color{incolor} }]:} \PY{c+c1}{\PYZsh{}\PYZhy{}\PYZhy{}\PYZhy{}\PYZhy{}\PYZhy{}\PYZhy{}\PYZhy{}\PYZhy{}\PYZhy{}\PYZhy{}\PYZhy{}\PYZhy{}\PYZhy{}\PYZhy{}\PYZhy{}\PYZhy{}\PYZhy{}\PYZhy{}\PYZhy{}\PYZhy{}\PYZhy{}\PYZhy{}\PYZhy{}\PYZhy{}\PYZhy{}\PYZhy{}\PYZhy{}\PYZhy{}\PYZhy{}\PYZhy{}\PYZhy{}\PYZhy{}\PYZhy{}\PYZhy{}\PYZhy{}\PYZhy{}\PYZhy{}\PYZhy{}\PYZhy{}\PYZhy{}\PYZhy{}\PYZhy{}\PYZhy{}\PYZhy{}\PYZhy{}\PYZhy{}\PYZhy{}\PYZhy{}\PYZhy{}\PYZhy{}\PYZhy{}\PYZhy{}\PYZhy{}\PYZhy{}\PYZhy{}\PYZhy{}\PYZhy{}\PYZhy{}\PYZhy{}\PYZhy{}}
        \PY{c+c1}{\PYZsh{} plot the RA/DEC in an area\PYZhy{}preserving projection}
        
        \PY{c+c1}{\PYZsh{}Actually fetch the sample from the Sloan data}
        \PY{n}{data} \PY{o}{=} \PY{n}{fetch\PYZus{}sdss\PYZus{}specgals}\PY{p}{(}\PY{p}{)}
        
        \PY{c+c1}{\PYZsh{} Define coordinate variables}
        \PY{n}{RA} \PY{o}{=} \PY{n}{data}\PY{p}{[}\PY{l+s+s1}{\PYZsq{}}\PY{l+s+s1}{ra}\PY{l+s+s1}{\PYZsq{}}\PY{p}{]}
        \PY{n}{DEC} \PY{o}{=} \PY{n}{data}\PY{p}{[}\PY{l+s+s1}{\PYZsq{}}\PY{l+s+s1}{dec}\PY{l+s+s1}{\PYZsq{}}\PY{p}{]}
        
        \PY{c+c1}{\PYZsh{} convert coordinates to degrees}
        \PY{n}{RA} \PY{o}{\PYZhy{}}\PY{o}{=} \PY{l+m+mi}{180}
        \PY{n}{RA} \PY{o}{*}\PY{o}{=} \PY{n}{np}\PY{o}{.}\PY{n}{pi} \PY{o}{/} \PY{l+m+mi}{180}
        \PY{n}{DEC} \PY{o}{*}\PY{o}{=} \PY{n}{np}\PY{o}{.}\PY{n}{pi} \PY{o}{/} \PY{l+m+mi}{180}
        
        \PY{c+c1}{\PYZsh{} keep galaxies in a selected area}
        \PY{c+c1}{\PYZsh{}rkeep = np.where(RA between [\PYZhy{}30,0] and DEC between [15,30])}
        
        \PY{n}{figure}\PY{p}{(}\PY{p}{)}
        \PY{n}{ax} \PY{o}{=} \PY{n}{plt}\PY{o}{.}\PY{n}{axes}\PY{p}{(}\PY{n}{projection}\PY{o}{=}\PY{l+s+s1}{\PYZsq{}}\PY{l+s+s1}{mollweide}\PY{l+s+s1}{\PYZsq{}}\PY{p}{)}
        
        \PY{n}{ax} \PY{o}{=} \PY{n}{plt}\PY{o}{.}\PY{n}{axes}\PY{p}{(}\PY{p}{)}
        \PY{n}{ax}\PY{o}{.}\PY{n}{grid}\PY{p}{(}\PY{p}{)}
        \PY{n}{plt}\PY{o}{.}\PY{n}{scatter}\PY{p}{(}\PY{n}{RA}\PY{p}{,} \PY{n}{DEC}\PY{p}{,} \PY{n}{s}\PY{o}{=}\PY{l+m+mi}{1}\PY{p}{,} \PY{n}{lw}\PY{o}{=}\PY{l+m+mi}{0}\PY{p}{,} \PY{n}{c}\PY{o}{=}\PY{n}{data}\PY{p}{[}\PY{l+s+s1}{\PYZsq{}}\PY{l+s+s1}{z}\PY{l+s+s1}{\PYZsq{}}\PY{p}{]}\PY{p}{,} \PY{n}{cmap}\PY{o}{=}\PY{n}{plt}\PY{o}{.}\PY{n}{cm}\PY{o}{.}\PY{n}{rainbow}\PY{p}{,}
                    \PY{n}{vmin}\PY{o}{=}\PY{l+m+mi}{0}\PY{p}{,} \PY{n}{vmax}\PY{o}{=}\PY{l+m+mf}{0.2}\PY{p}{)}
        
        \PY{n}{plt}\PY{o}{.}\PY{n}{title}\PY{p}{(}\PY{l+s+s1}{\PYZsq{}}\PY{l+s+s1}{SDSS DR8 Spectroscopic Galaxies}\PY{l+s+s1}{\PYZsq{}}\PY{p}{)}
        \PY{n}{cb} \PY{o}{=} \PY{n}{plt}\PY{o}{.}\PY{n}{colorbar}\PY{p}{(}\PY{n}{cax}\PY{o}{=}\PY{n}{plt}\PY{o}{.}\PY{n}{axes}\PY{p}{(}\PY{p}{[}\PY{l+m+mf}{0.05}\PY{p}{,} \PY{l+m+mf}{0.1}\PY{p}{,} \PY{l+m+mf}{0.9}\PY{p}{,} \PY{l+m+mf}{0.05}\PY{p}{]}\PY{p}{)}\PY{p}{,}
                          \PY{n}{orientation}\PY{o}{=}\PY{l+s+s1}{\PYZsq{}}\PY{l+s+s1}{horizontal}\PY{l+s+s1}{\PYZsq{}}\PY{p}{,}
                          \PY{n}{ticks}\PY{o}{=}\PY{n}{np}\PY{o}{.}\PY{n}{linspace}\PY{p}{(}\PY{l+m+mi}{0}\PY{p}{,} \PY{l+m+mf}{0.2}\PY{p}{,} \PY{l+m+mi}{9}\PY{p}{)}\PY{p}{)}
        \PY{n}{cb}\PY{o}{.}\PY{n}{set\PYZus{}label}\PY{p}{(}\PY{l+s+s1}{\PYZsq{}}\PY{l+s+s1}{redshift}\PY{l+s+s1}{\PYZsq{}}\PY{p}{)}
\end{Verbatim}


    HINT: for more color maps, you can look at this reference page. For
example, you can replace "rainbow" with "autumn\_r".


    % Add a bibliography block to the postdoc
    
    
    
    \end{document}
