
% Default to the notebook output style

    


% Inherit from the specified cell style.




    
\documentclass[11pt]{article}

    
    
    \usepackage[T1]{fontenc}
    % Nicer default font (+ math font) than Computer Modern for most use cases
    \usepackage{mathpazo}

    % Basic figure setup, for now with no caption control since it's done
    % automatically by Pandoc (which extracts ![](path) syntax from Markdown).
    \usepackage{graphicx}
    % We will generate all images so they have a width \maxwidth. This means
    % that they will get their normal width if they fit onto the page, but
    % are scaled down if they would overflow the margins.
    \makeatletter
    \def\maxwidth{\ifdim\Gin@nat@width>\linewidth\linewidth
    \else\Gin@nat@width\fi}
    \makeatother
    \let\Oldincludegraphics\includegraphics
    % Set max figure width to be 80% of text width, for now hardcoded.
    \renewcommand{\includegraphics}[1]{\Oldincludegraphics[width=.8\maxwidth]{#1}}
    % Ensure that by default, figures have no caption (until we provide a
    % proper Figure object with a Caption API and a way to capture that
    % in the conversion process - todo).
    \usepackage{caption}
    \DeclareCaptionLabelFormat{nolabel}{}
    \captionsetup{labelformat=nolabel}

    \usepackage{adjustbox} % Used to constrain images to a maximum size 
    \usepackage{xcolor} % Allow colors to be defined
    \usepackage{enumerate} % Needed for markdown enumerations to work
    \usepackage{geometry} % Used to adjust the document margins
    \usepackage{amsmath} % Equations
    \usepackage{amssymb} % Equations
    \usepackage{textcomp} % defines textquotesingle
    % Hack from http://tex.stackexchange.com/a/47451/13684:
    \AtBeginDocument{%
        \def\PYZsq{\textquotesingle}% Upright quotes in Pygmentized code
    }
    \usepackage{upquote} % Upright quotes for verbatim code
    \usepackage{eurosym} % defines \euro
    \usepackage[mathletters]{ucs} % Extended unicode (utf-8) support
    \usepackage[utf8x]{inputenc} % Allow utf-8 characters in the tex document
    \usepackage{fancyvrb} % verbatim replacement that allows latex
    \usepackage{grffile} % extends the file name processing of package graphics 
                         % to support a larger range 
    % The hyperref package gives us a pdf with properly built
    % internal navigation ('pdf bookmarks' for the table of contents,
    % internal cross-reference links, web links for URLs, etc.)
    \usepackage{hyperref}
    \usepackage{longtable} % longtable support required by pandoc >1.10
    \usepackage{booktabs}  % table support for pandoc > 1.12.2
    \usepackage[inline]{enumitem} % IRkernel/repr support (it uses the enumerate* environment)
    \usepackage[normalem]{ulem} % ulem is needed to support strikethroughs (\sout)
                                % normalem makes italics be italics, not underlines
    

    
    
    % Colors for the hyperref package
    \definecolor{urlcolor}{rgb}{0,.145,.698}
    \definecolor{linkcolor}{rgb}{.71,0.21,0.01}
    \definecolor{citecolor}{rgb}{.12,.54,.11}

    % ANSI colors
    \definecolor{ansi-black}{HTML}{3E424D}
    \definecolor{ansi-black-intense}{HTML}{282C36}
    \definecolor{ansi-red}{HTML}{E75C58}
    \definecolor{ansi-red-intense}{HTML}{B22B31}
    \definecolor{ansi-green}{HTML}{00A250}
    \definecolor{ansi-green-intense}{HTML}{007427}
    \definecolor{ansi-yellow}{HTML}{DDB62B}
    \definecolor{ansi-yellow-intense}{HTML}{B27D12}
    \definecolor{ansi-blue}{HTML}{208FFB}
    \definecolor{ansi-blue-intense}{HTML}{0065CA}
    \definecolor{ansi-magenta}{HTML}{D160C4}
    \definecolor{ansi-magenta-intense}{HTML}{A03196}
    \definecolor{ansi-cyan}{HTML}{60C6C8}
    \definecolor{ansi-cyan-intense}{HTML}{258F8F}
    \definecolor{ansi-white}{HTML}{C5C1B4}
    \definecolor{ansi-white-intense}{HTML}{A1A6B2}

    % commands and environments needed by pandoc snippets
    % extracted from the output of `pandoc -s`
    \providecommand{\tightlist}{%
      \setlength{\itemsep}{0pt}\setlength{\parskip}{0pt}}
    \DefineVerbatimEnvironment{Highlighting}{Verbatim}{commandchars=\\\{\}}
    % Add ',fontsize=\small' for more characters per line
    \newenvironment{Shaded}{}{}
    \newcommand{\KeywordTok}[1]{\textcolor[rgb]{0.00,0.44,0.13}{\textbf{{#1}}}}
    \newcommand{\DataTypeTok}[1]{\textcolor[rgb]{0.56,0.13,0.00}{{#1}}}
    \newcommand{\DecValTok}[1]{\textcolor[rgb]{0.25,0.63,0.44}{{#1}}}
    \newcommand{\BaseNTok}[1]{\textcolor[rgb]{0.25,0.63,0.44}{{#1}}}
    \newcommand{\FloatTok}[1]{\textcolor[rgb]{0.25,0.63,0.44}{{#1}}}
    \newcommand{\CharTok}[1]{\textcolor[rgb]{0.25,0.44,0.63}{{#1}}}
    \newcommand{\StringTok}[1]{\textcolor[rgb]{0.25,0.44,0.63}{{#1}}}
    \newcommand{\CommentTok}[1]{\textcolor[rgb]{0.38,0.63,0.69}{\textit{{#1}}}}
    \newcommand{\OtherTok}[1]{\textcolor[rgb]{0.00,0.44,0.13}{{#1}}}
    \newcommand{\AlertTok}[1]{\textcolor[rgb]{1.00,0.00,0.00}{\textbf{{#1}}}}
    \newcommand{\FunctionTok}[1]{\textcolor[rgb]{0.02,0.16,0.49}{{#1}}}
    \newcommand{\RegionMarkerTok}[1]{{#1}}
    \newcommand{\ErrorTok}[1]{\textcolor[rgb]{1.00,0.00,0.00}{\textbf{{#1}}}}
    \newcommand{\NormalTok}[1]{{#1}}
    
    % Additional commands for more recent versions of Pandoc
    \newcommand{\ConstantTok}[1]{\textcolor[rgb]{0.53,0.00,0.00}{{#1}}}
    \newcommand{\SpecialCharTok}[1]{\textcolor[rgb]{0.25,0.44,0.63}{{#1}}}
    \newcommand{\VerbatimStringTok}[1]{\textcolor[rgb]{0.25,0.44,0.63}{{#1}}}
    \newcommand{\SpecialStringTok}[1]{\textcolor[rgb]{0.73,0.40,0.53}{{#1}}}
    \newcommand{\ImportTok}[1]{{#1}}
    \newcommand{\DocumentationTok}[1]{\textcolor[rgb]{0.73,0.13,0.13}{\textit{{#1}}}}
    \newcommand{\AnnotationTok}[1]{\textcolor[rgb]{0.38,0.63,0.69}{\textbf{\textit{{#1}}}}}
    \newcommand{\CommentVarTok}[1]{\textcolor[rgb]{0.38,0.63,0.69}{\textbf{\textit{{#1}}}}}
    \newcommand{\VariableTok}[1]{\textcolor[rgb]{0.10,0.09,0.49}{{#1}}}
    \newcommand{\ControlFlowTok}[1]{\textcolor[rgb]{0.00,0.44,0.13}{\textbf{{#1}}}}
    \newcommand{\OperatorTok}[1]{\textcolor[rgb]{0.40,0.40,0.40}{{#1}}}
    \newcommand{\BuiltInTok}[1]{{#1}}
    \newcommand{\ExtensionTok}[1]{{#1}}
    \newcommand{\PreprocessorTok}[1]{\textcolor[rgb]{0.74,0.48,0.00}{{#1}}}
    \newcommand{\AttributeTok}[1]{\textcolor[rgb]{0.49,0.56,0.16}{{#1}}}
    \newcommand{\InformationTok}[1]{\textcolor[rgb]{0.38,0.63,0.69}{\textbf{\textit{{#1}}}}}
    \newcommand{\WarningTok}[1]{\textcolor[rgb]{0.38,0.63,0.69}{\textbf{\textit{{#1}}}}}
    
    
    % Define a nice break command that doesn't care if a line doesn't already
    % exist.
    \def\br{\hspace*{\fill} \\* }
    % Math Jax compatability definitions
    \def\gt{>}
    \def\lt{<}
    % Document parameters
    \title{Lives\_and\_Deaths\_of\_Stars}
    
    
    

    % Pygments definitions
    
\makeatletter
\def\PY@reset{\let\PY@it=\relax \let\PY@bf=\relax%
    \let\PY@ul=\relax \let\PY@tc=\relax%
    \let\PY@bc=\relax \let\PY@ff=\relax}
\def\PY@tok#1{\csname PY@tok@#1\endcsname}
\def\PY@toks#1+{\ifx\relax#1\empty\else%
    \PY@tok{#1}\expandafter\PY@toks\fi}
\def\PY@do#1{\PY@bc{\PY@tc{\PY@ul{%
    \PY@it{\PY@bf{\PY@ff{#1}}}}}}}
\def\PY#1#2{\PY@reset\PY@toks#1+\relax+\PY@do{#2}}

\expandafter\def\csname PY@tok@w\endcsname{\def\PY@tc##1{\textcolor[rgb]{0.73,0.73,0.73}{##1}}}
\expandafter\def\csname PY@tok@c\endcsname{\let\PY@it=\textit\def\PY@tc##1{\textcolor[rgb]{0.25,0.50,0.50}{##1}}}
\expandafter\def\csname PY@tok@cp\endcsname{\def\PY@tc##1{\textcolor[rgb]{0.74,0.48,0.00}{##1}}}
\expandafter\def\csname PY@tok@k\endcsname{\let\PY@bf=\textbf\def\PY@tc##1{\textcolor[rgb]{0.00,0.50,0.00}{##1}}}
\expandafter\def\csname PY@tok@kp\endcsname{\def\PY@tc##1{\textcolor[rgb]{0.00,0.50,0.00}{##1}}}
\expandafter\def\csname PY@tok@kt\endcsname{\def\PY@tc##1{\textcolor[rgb]{0.69,0.00,0.25}{##1}}}
\expandafter\def\csname PY@tok@o\endcsname{\def\PY@tc##1{\textcolor[rgb]{0.40,0.40,0.40}{##1}}}
\expandafter\def\csname PY@tok@ow\endcsname{\let\PY@bf=\textbf\def\PY@tc##1{\textcolor[rgb]{0.67,0.13,1.00}{##1}}}
\expandafter\def\csname PY@tok@nb\endcsname{\def\PY@tc##1{\textcolor[rgb]{0.00,0.50,0.00}{##1}}}
\expandafter\def\csname PY@tok@nf\endcsname{\def\PY@tc##1{\textcolor[rgb]{0.00,0.00,1.00}{##1}}}
\expandafter\def\csname PY@tok@nc\endcsname{\let\PY@bf=\textbf\def\PY@tc##1{\textcolor[rgb]{0.00,0.00,1.00}{##1}}}
\expandafter\def\csname PY@tok@nn\endcsname{\let\PY@bf=\textbf\def\PY@tc##1{\textcolor[rgb]{0.00,0.00,1.00}{##1}}}
\expandafter\def\csname PY@tok@ne\endcsname{\let\PY@bf=\textbf\def\PY@tc##1{\textcolor[rgb]{0.82,0.25,0.23}{##1}}}
\expandafter\def\csname PY@tok@nv\endcsname{\def\PY@tc##1{\textcolor[rgb]{0.10,0.09,0.49}{##1}}}
\expandafter\def\csname PY@tok@no\endcsname{\def\PY@tc##1{\textcolor[rgb]{0.53,0.00,0.00}{##1}}}
\expandafter\def\csname PY@tok@nl\endcsname{\def\PY@tc##1{\textcolor[rgb]{0.63,0.63,0.00}{##1}}}
\expandafter\def\csname PY@tok@ni\endcsname{\let\PY@bf=\textbf\def\PY@tc##1{\textcolor[rgb]{0.60,0.60,0.60}{##1}}}
\expandafter\def\csname PY@tok@na\endcsname{\def\PY@tc##1{\textcolor[rgb]{0.49,0.56,0.16}{##1}}}
\expandafter\def\csname PY@tok@nt\endcsname{\let\PY@bf=\textbf\def\PY@tc##1{\textcolor[rgb]{0.00,0.50,0.00}{##1}}}
\expandafter\def\csname PY@tok@nd\endcsname{\def\PY@tc##1{\textcolor[rgb]{0.67,0.13,1.00}{##1}}}
\expandafter\def\csname PY@tok@s\endcsname{\def\PY@tc##1{\textcolor[rgb]{0.73,0.13,0.13}{##1}}}
\expandafter\def\csname PY@tok@sd\endcsname{\let\PY@it=\textit\def\PY@tc##1{\textcolor[rgb]{0.73,0.13,0.13}{##1}}}
\expandafter\def\csname PY@tok@si\endcsname{\let\PY@bf=\textbf\def\PY@tc##1{\textcolor[rgb]{0.73,0.40,0.53}{##1}}}
\expandafter\def\csname PY@tok@se\endcsname{\let\PY@bf=\textbf\def\PY@tc##1{\textcolor[rgb]{0.73,0.40,0.13}{##1}}}
\expandafter\def\csname PY@tok@sr\endcsname{\def\PY@tc##1{\textcolor[rgb]{0.73,0.40,0.53}{##1}}}
\expandafter\def\csname PY@tok@ss\endcsname{\def\PY@tc##1{\textcolor[rgb]{0.10,0.09,0.49}{##1}}}
\expandafter\def\csname PY@tok@sx\endcsname{\def\PY@tc##1{\textcolor[rgb]{0.00,0.50,0.00}{##1}}}
\expandafter\def\csname PY@tok@m\endcsname{\def\PY@tc##1{\textcolor[rgb]{0.40,0.40,0.40}{##1}}}
\expandafter\def\csname PY@tok@gh\endcsname{\let\PY@bf=\textbf\def\PY@tc##1{\textcolor[rgb]{0.00,0.00,0.50}{##1}}}
\expandafter\def\csname PY@tok@gu\endcsname{\let\PY@bf=\textbf\def\PY@tc##1{\textcolor[rgb]{0.50,0.00,0.50}{##1}}}
\expandafter\def\csname PY@tok@gd\endcsname{\def\PY@tc##1{\textcolor[rgb]{0.63,0.00,0.00}{##1}}}
\expandafter\def\csname PY@tok@gi\endcsname{\def\PY@tc##1{\textcolor[rgb]{0.00,0.63,0.00}{##1}}}
\expandafter\def\csname PY@tok@gr\endcsname{\def\PY@tc##1{\textcolor[rgb]{1.00,0.00,0.00}{##1}}}
\expandafter\def\csname PY@tok@ge\endcsname{\let\PY@it=\textit}
\expandafter\def\csname PY@tok@gs\endcsname{\let\PY@bf=\textbf}
\expandafter\def\csname PY@tok@gp\endcsname{\let\PY@bf=\textbf\def\PY@tc##1{\textcolor[rgb]{0.00,0.00,0.50}{##1}}}
\expandafter\def\csname PY@tok@go\endcsname{\def\PY@tc##1{\textcolor[rgb]{0.53,0.53,0.53}{##1}}}
\expandafter\def\csname PY@tok@gt\endcsname{\def\PY@tc##1{\textcolor[rgb]{0.00,0.27,0.87}{##1}}}
\expandafter\def\csname PY@tok@err\endcsname{\def\PY@bc##1{\setlength{\fboxsep}{0pt}\fcolorbox[rgb]{1.00,0.00,0.00}{1,1,1}{\strut ##1}}}
\expandafter\def\csname PY@tok@kc\endcsname{\let\PY@bf=\textbf\def\PY@tc##1{\textcolor[rgb]{0.00,0.50,0.00}{##1}}}
\expandafter\def\csname PY@tok@kd\endcsname{\let\PY@bf=\textbf\def\PY@tc##1{\textcolor[rgb]{0.00,0.50,0.00}{##1}}}
\expandafter\def\csname PY@tok@kn\endcsname{\let\PY@bf=\textbf\def\PY@tc##1{\textcolor[rgb]{0.00,0.50,0.00}{##1}}}
\expandafter\def\csname PY@tok@kr\endcsname{\let\PY@bf=\textbf\def\PY@tc##1{\textcolor[rgb]{0.00,0.50,0.00}{##1}}}
\expandafter\def\csname PY@tok@bp\endcsname{\def\PY@tc##1{\textcolor[rgb]{0.00,0.50,0.00}{##1}}}
\expandafter\def\csname PY@tok@fm\endcsname{\def\PY@tc##1{\textcolor[rgb]{0.00,0.00,1.00}{##1}}}
\expandafter\def\csname PY@tok@vc\endcsname{\def\PY@tc##1{\textcolor[rgb]{0.10,0.09,0.49}{##1}}}
\expandafter\def\csname PY@tok@vg\endcsname{\def\PY@tc##1{\textcolor[rgb]{0.10,0.09,0.49}{##1}}}
\expandafter\def\csname PY@tok@vi\endcsname{\def\PY@tc##1{\textcolor[rgb]{0.10,0.09,0.49}{##1}}}
\expandafter\def\csname PY@tok@vm\endcsname{\def\PY@tc##1{\textcolor[rgb]{0.10,0.09,0.49}{##1}}}
\expandafter\def\csname PY@tok@sa\endcsname{\def\PY@tc##1{\textcolor[rgb]{0.73,0.13,0.13}{##1}}}
\expandafter\def\csname PY@tok@sb\endcsname{\def\PY@tc##1{\textcolor[rgb]{0.73,0.13,0.13}{##1}}}
\expandafter\def\csname PY@tok@sc\endcsname{\def\PY@tc##1{\textcolor[rgb]{0.73,0.13,0.13}{##1}}}
\expandafter\def\csname PY@tok@dl\endcsname{\def\PY@tc##1{\textcolor[rgb]{0.73,0.13,0.13}{##1}}}
\expandafter\def\csname PY@tok@s2\endcsname{\def\PY@tc##1{\textcolor[rgb]{0.73,0.13,0.13}{##1}}}
\expandafter\def\csname PY@tok@sh\endcsname{\def\PY@tc##1{\textcolor[rgb]{0.73,0.13,0.13}{##1}}}
\expandafter\def\csname PY@tok@s1\endcsname{\def\PY@tc##1{\textcolor[rgb]{0.73,0.13,0.13}{##1}}}
\expandafter\def\csname PY@tok@mb\endcsname{\def\PY@tc##1{\textcolor[rgb]{0.40,0.40,0.40}{##1}}}
\expandafter\def\csname PY@tok@mf\endcsname{\def\PY@tc##1{\textcolor[rgb]{0.40,0.40,0.40}{##1}}}
\expandafter\def\csname PY@tok@mh\endcsname{\def\PY@tc##1{\textcolor[rgb]{0.40,0.40,0.40}{##1}}}
\expandafter\def\csname PY@tok@mi\endcsname{\def\PY@tc##1{\textcolor[rgb]{0.40,0.40,0.40}{##1}}}
\expandafter\def\csname PY@tok@il\endcsname{\def\PY@tc##1{\textcolor[rgb]{0.40,0.40,0.40}{##1}}}
\expandafter\def\csname PY@tok@mo\endcsname{\def\PY@tc##1{\textcolor[rgb]{0.40,0.40,0.40}{##1}}}
\expandafter\def\csname PY@tok@ch\endcsname{\let\PY@it=\textit\def\PY@tc##1{\textcolor[rgb]{0.25,0.50,0.50}{##1}}}
\expandafter\def\csname PY@tok@cm\endcsname{\let\PY@it=\textit\def\PY@tc##1{\textcolor[rgb]{0.25,0.50,0.50}{##1}}}
\expandafter\def\csname PY@tok@cpf\endcsname{\let\PY@it=\textit\def\PY@tc##1{\textcolor[rgb]{0.25,0.50,0.50}{##1}}}
\expandafter\def\csname PY@tok@c1\endcsname{\let\PY@it=\textit\def\PY@tc##1{\textcolor[rgb]{0.25,0.50,0.50}{##1}}}
\expandafter\def\csname PY@tok@cs\endcsname{\let\PY@it=\textit\def\PY@tc##1{\textcolor[rgb]{0.25,0.50,0.50}{##1}}}

\def\PYZbs{\char`\\}
\def\PYZus{\char`\_}
\def\PYZob{\char`\{}
\def\PYZcb{\char`\}}
\def\PYZca{\char`\^}
\def\PYZam{\char`\&}
\def\PYZlt{\char`\<}
\def\PYZgt{\char`\>}
\def\PYZsh{\char`\#}
\def\PYZpc{\char`\%}
\def\PYZdl{\char`\$}
\def\PYZhy{\char`\-}
\def\PYZsq{\char`\'}
\def\PYZdq{\char`\"}
\def\PYZti{\char`\~}
% for compatibility with earlier versions
\def\PYZat{@}
\def\PYZlb{[}
\def\PYZrb{]}
\makeatother


    % Exact colors from NB
    \definecolor{incolor}{rgb}{0.0, 0.0, 0.5}
    \definecolor{outcolor}{rgb}{0.545, 0.0, 0.0}



    
    % Prevent overflowing lines due to hard-to-break entities
    \sloppy 
    % Setup hyperref package
    \hypersetup{
      breaklinks=true,  % so long urls are correctly broken across lines
      colorlinks=true,
      urlcolor=urlcolor,
      linkcolor=linkcolor,
      citecolor=citecolor,
      }
    % Slightly bigger margins than the latex defaults
    
    \geometry{verbose,tmargin=1in,bmargin=1in,lmargin=1in,rmargin=1in}
    
    

    \begin{document}
    
    
    \maketitle
    
    

    
     The Lives and Deaths of Stars

\subparagraph{Written by Gautham Narayan,
STScI}\label{written-by-gautham-narayan-stsci}

We talked about the lives of deaths and stars, but now it's time to play
with real data, and see this story with your own eyes. This is
interactive! You get to do anything you like with the data! DON'T WORRY!
YOU WILL NOT BREAK ANYTHING! Want to try something? Just go right ahead!

Want to know how to do something but aren't sure? Ask questions! \_\_\_

    \section{Table of Contents}\label{table-of-contents}

\begin{itemize}
\tightlist
\item
  Section \ref{how-to-use-this-notebook}
\item
  Section \ref{pre-activity-setup}
\item
  Section \ref{activity-1-graphing-some-data}

  \begin{itemize}
  \tightlist
  \item
    Section \ref{step-11-making-a-scatter-plot}
  \item
    Section \ref{step-12-graphing-a-function}
  \end{itemize}
\item
  Section \ref{activity-2-graphing-real-data-from-stars}

  \begin{itemize}
  \tightlist
  \item
    Section \ref{step-21-selecting-spectra-to-plot}
  \item
    Section \ref{step-22-plotting-your-first-spectrum}
  \item
    Section \ref{step-23-plotting-a-second-spectrum}
  \item
    Section \ref{step-24-plotting-vega}
  \end{itemize}
\item
  Section \ref{activity-3-plotting-the-main-sequence}

  \begin{itemize}
  \tightlist
  \item
    Section \ref{step-31-entering-the-data-from-some-stars}
  \item
    Section \ref{step-32-plot-the-star-data}
  \item
    Section \ref{step-33-get-a-sense-for-the-sizes-and-colors-of-these-stars}
  \end{itemize}
\item
  Section \ref{activity-4-looking-at-the-life-cycles-of-stars} \_\_\_
\end{itemize}

    \section{How to Use This Notebook}\label{how-to-use-this-notebook}

The webpage you are in is actually an app - much like the ones on your
cellphone. This app consists of cells.

An \emph{input} cell looks like a light grey box with an
\texttt{In\ {[}\ {]}:} on its left. Input cells each contain code -
instructions to make the computer do something.

To activate or select a cell, click anywhere inside of it.

\begin{verbatim}
<font size='3'><b>Select the cell below and read its contents.</b></font>
\end{verbatim}

    \begin{Verbatim}[commandchars=\\\{\}]
{\color{incolor}In [{\color{incolor} }]:} \PY{c+c1}{\PYZsh{} Text that follows a \PYZdq{}\PYZsh{}\PYZdq{} is known as a comment.}
        \PY{c+c1}{\PYZsh{} Comments do not affect your code in any way.}
        \PY{c+c1}{\PYZsh{} You should always read the comments at the top of each cell you interact with.}
        \PY{c+c1}{\PYZsh{} Comments will be used to describe what the cell\PYZsq{}s code is actually doing.}
\end{Verbatim}


    To execute or run a selected cell, hit \texttt{{[}Shift\ +\ Enter{]}} on
your keyboard.

\begin{verbatim}
<font size='3'><b>Select the cell below and read its contents. Then, run the cell.</b></font>
\end{verbatim}

    \begin{Verbatim}[commandchars=\\\{\}]
{\color{incolor}In [{\color{incolor} }]:} \PY{c+c1}{\PYZsh{} Text that DOESN\PYZsq{}T follow a \PYZdq{}\PYZsh{}\PYZdq{} is considered code.}
        \PY{c+c1}{\PYZsh{} Lines of code are instructions given to your computer.}
        \PY{c+c1}{\PYZsh{} The line of code below is a \PYZdq{}print\PYZdq{} statement.}
        \PY{c+c1}{\PYZsh{} A print statement literally prints out the text between its quotes.}
        
        \PY{n+nb}{print}\PY{p}{(}\PY{l+s+s2}{\PYZdq{}}\PY{l+s+s2}{Congrats! You have successfully run your first cell!}\PY{l+s+s2}{\PYZdq{}}\PY{p}{)}
\end{Verbatim}


    Running a cell creates an \emph{output} directly below it. An output can
be some text, a graph, an interactive slider, or even nothing at all!
For that last case, you know you have run a cell when the
\texttt{In\ {[}\ {]}:} becomes \texttt{In\ {[}\#{]}:}, where "\#" is any
number.

You can learn more about how Jupyter notebooks work at
https://try.jupyter.org/ \_\_\_

    \section{Pre-Activity Setup}\label{pre-activity-setup}

In order for any of the activities to work properly, you must import the
libraries needed for the code in this notebook.

\begin{verbatim}
<font size='3'><b>Select and run the cell below.</b></font>
\end{verbatim}

    \begin{Verbatim}[commandchars=\\\{\}]
{\color{incolor}In [{\color{incolor} }]:} \PY{c+c1}{\PYZsh{} Here, you are importing the libraries needed for this notebook.}
        \PY{c+c1}{\PYZsh{} These libraries set up the plotting environment in your browser.}
        
        \PY{o}{\PYZpc{}}\PY{k}{matplotlib} notebook
        \PY{o}{\PYZpc{}}\PY{k}{pylab}
        \PY{k+kn}{from} \PY{n+nn}{astroML}\PY{n+nn}{.}\PY{n+nn}{datasets} \PY{k}{import} \PY{n}{fetch\PYZus{}sdss\PYZus{}spectrum}\PY{p}{,} \PY{n}{fetch\PYZus{}vega\PYZus{}spectrum}\PY{p}{,} \PY{n}{fetch\PYZus{}sdss\PYZus{}S82standards}
        \PY{k+kn}{from} \PY{n+nn}{astroML}\PY{n+nn}{.}\PY{n+nn}{plotting} \PY{k}{import} \PY{n}{MultiAxes}
        \PY{k+kn}{from} \PY{n+nn}{IPython}\PY{n+nn}{.}\PY{n+nn}{core}\PY{n+nn}{.}\PY{n+nn}{display} \PY{k}{import} \PY{n}{Image}\PY{p}{,} \PY{n}{display}
\end{Verbatim}


    \begin{center}\rule{0.5\linewidth}{\linethickness}\end{center}

    \section{Activity 1: Graphing Some
Data}\label{activity-1-graphing-some-data}

In this activity, you'll take the first step towards become a computer
programmer! One of the first things scientists want to do when they get
data is to take a look at it - plot it on a graph! This activity is
going to show you how to plot data. \_\_\_

    \subsection{Step 1.1: Making a Scatter
Plot}\label{step-1.1-making-a-scatter-plot}

\begin{verbatim}
<h3 class='alert-heading'>Helpful Reminder(s)</h3>
<ul>
    <li>Click anywhere inside of a cell to select it.</li>
    <li>Hit [Shift + Enter] to run a selected cell.</li>
</ul>
\end{verbatim}

    \begin{Verbatim}[commandchars=\\\{\}]
{\color{incolor}In [{\color{incolor} }]:} \PY{c+c1}{\PYZsh{} You don\PYZsq{}t need to use the same numbers, or indeed even these many \PYZhy{} put in whatever you like for x \PYZam{} y.}
        
        \PY{n}{x} \PY{o}{=} \PY{p}{[}\PY{l+m+mi}{11}\PY{p}{,} \PY{l+m+mi}{31}\PY{p}{,} \PY{l+m+mi}{2}\PY{p}{,} \PY{o}{\PYZhy{}}\PY{l+m+mi}{7}\PY{p}{,} \PY{l+m+mi}{4}\PY{p}{,} \PY{o}{\PYZhy{}}\PY{l+m+mi}{10}\PY{p}{,} \PY{l+m+mi}{6}\PY{p}{]}
        \PY{n}{y} \PY{o}{=} \PY{p}{[}\PY{l+m+mi}{10}\PY{p}{,} \PY{l+m+mi}{21}\PY{p}{,} \PY{l+m+mi}{0}\PY{p}{,} \PY{l+m+mi}{9}\PY{p}{,} \PY{l+m+mi}{17}\PY{p}{,} \PY{l+m+mi}{13}\PY{p}{,} \PY{l+m+mi}{18}\PY{p}{]}
        \PY{n}{figure}\PY{p}{(}\PY{p}{)}
        \PY{n}{scatter}\PY{p}{(}\PY{n}{x}\PY{p}{,}\PY{n}{y}\PY{p}{,}\PY{n}{marker}\PY{o}{=}\PY{l+s+s1}{\PYZsq{}}\PY{l+s+s1}{o}\PY{l+s+s1}{\PYZsq{}}\PY{p}{,}\PY{n}{color}\PY{o}{=}\PY{l+s+s1}{\PYZsq{}}\PY{l+s+s1}{blue}\PY{l+s+s1}{\PYZsq{}}\PY{p}{)}
        \PY{n}{xlabel}\PY{p}{(}\PY{l+s+s1}{\PYZsq{}}\PY{l+s+s1}{X}\PY{l+s+s1}{\PYZsq{}}\PY{p}{)}
        \PY{n}{ylabel}\PY{p}{(}\PY{l+s+s1}{\PYZsq{}}\PY{l+s+s1}{Y}\PY{l+s+s1}{\PYZsq{}}\PY{p}{)}
\end{Verbatim}


    Scatter is the most basic plotting command, and plots a bunch of point -
notice that there are blue dots on the X and Y locations you entered!
\_\_\_

    \subsection{Step 1.2: Graphing a
Function}\label{step-1.2-graphing-a-function}

There are several other plotting commands - the one we'll use mostly is
called "plot".

    \begin{Verbatim}[commandchars=\\\{\}]
{\color{incolor}In [{\color{incolor} }]:} \PY{c+c1}{\PYZsh{} What\PYZsq{}s nifty is that you can plot functions as well! }
        \PY{c+c1}{\PYZsh{} Want to know what x to the 3.5 power + 57 is? Compute and plot it!}
        
        \PY{c+c1}{\PYZsh{} This command creates a series of numbers from 0 to 20 increasing by 1.}
        \PY{n}{x} \PY{o}{=} \PY{n}{linspace}\PY{p}{(}\PY{o}{\PYZhy{}}\PY{l+m+mi}{90}\PY{p}{,}\PY{l+m+mi}{90}\PY{p}{,}\PY{l+m+mi}{20000}\PY{p}{)}
        
        \PY{c+c1}{\PYZsh{} If you want, you can print it out.}
        \PY{n+nb}{print}\PY{p}{(}\PY{n}{x}\PY{p}{)} 
        
        \PY{c+c1}{\PYZsh{} You can now relate y to x with some function of the sort you\PYZsq{}ve learned in school.}
        \PY{n}{y} \PY{o}{=} \PY{l+m+mi}{3}\PY{o}{*}\PY{n}{x} \PY{o}{+} \PY{l+m+mf}{5.1} \PY{o}{+} \PY{l+m+mi}{40}\PY{o}{*}\PY{n}{sin}\PY{p}{(}\PY{n}{x}\PY{o}{/}\PY{l+m+mf}{10.}\PY{p}{)} \PY{o}{+} \PY{l+m+mi}{20}\PY{o}{*}\PY{n}{tan}\PY{p}{(}\PY{n}{x}\PY{p}{)}
        
        \PY{c+c1}{\PYZsh{} Now, you can plot your data!}
        \PY{n}{figure}\PY{p}{(}\PY{p}{)}
        \PY{n}{plot}\PY{p}{(}\PY{n}{x}\PY{p}{,}\PY{n}{y}\PY{p}{,} \PY{n}{marker}\PY{o}{=}\PY{l+s+s1}{\PYZsq{}}\PY{l+s+s1}{None}\PY{l+s+s1}{\PYZsq{}}\PY{p}{,} \PY{n}{color}\PY{o}{=}\PY{l+s+s1}{\PYZsq{}}\PY{l+s+s1}{green}\PY{l+s+s1}{\PYZsq{}}\PY{p}{,} \PY{n}{linestyle}\PY{o}{=}\PY{l+s+s1}{\PYZsq{}}\PY{l+s+s1}{\PYZhy{}}\PY{l+s+s1}{\PYZsq{}}\PY{p}{)}
        \PY{n}{xlabel}\PY{p}{(}\PY{l+s+s1}{\PYZsq{}}\PY{l+s+s1}{X}\PY{l+s+s1}{\PYZsq{}}\PY{p}{)}
        \PY{n}{ylabel}\PY{p}{(}\PY{l+s+s1}{\PYZsq{}}\PY{l+s+s1}{Y}\PY{l+s+s1}{\PYZsq{}}\PY{p}{)}
\end{Verbatim}


    Nifty! We've gone from plotting a bunch of points to plotting something
continious - a curve.

You can plot all sorts of different curves - parabolas, sines and
cosines or other trig functions, or whatever you can imagine really! Try
it! You can create new cells below this one, by clicking it, then using
the menu at the top - \textbf{Insert \textgreater{} Insert Cell Below}.

Type in your code for y based on x (you can even change x) and then plot
it with the plot command from above. Then, hit
\texttt{{[}Shift\ +\ Enter{]}} to run it! \_\_\_

    \section{Activity 2: Graphing Real Data from
Stars}\label{activity-2-graphing-real-data-from-stars}

Just like we plotted the data above, we can now plot a spectrum of a
star!

To do this, we are going to use data from the
\href{http://sdss.org}{Sloan Digital Sky Survey (SDSS)}. This project
used a telescope at Apache Point in New Mexico to look at the northern
sky.

\begin{verbatim}
<center>
    <br>
    <img src='https://apod.nasa.gov/apod/image/9806/sloan_fermilab_big.jpg', width='300'>
    <br>
    <figcaption>
        <font color='grey'>
            <b>Figure 1:</b>
            The Sloan Telescope at Apache Point, New Mexico.
            <br>
            <b>Image Credit:</b>
            SDSS Team, Fermilab Visual Media Services.
        </font>
    </figcaption>
</center>
\end{verbatim}

It was the first "Big Data" project in astronomy. Sloan found millions
of stars and galaxies, and made their data public. What we're going to
do is start to play with SDSS data. \_\_\_

    \subsection{Step 2.1: Selecting Spectra to
Plot}\label{step-2.1-selecting-spectra-to-plot}

Click \href{http://classic.sdss.org/dr5/algorithms/spectemplates/}{here}
for the spectra of different stars.

Look at the gif links, and look for \textbf{Plate}, \textbf{Fiber} and
\textbf{MJD}. Write them down on a piece of paper.

Next, let's go look at the data for that star
\href{http://cas.sdss.org/dr14/en/tools/explore/Summary.aspx}{here}.

Click on \textbf{Search} on the left hand side menu bar, and then enter
the \textbf{Plate}, \textbf{Fiber} and \textbf{MJD} that you wrote down
there, and hit \textbf{Go}.

If you click on the image, you can move around, zoom in and out - it's
like Google Maps for the night sky! \_\_\_

    \subsection{Step 2.2: Plotting Your First
Spectrum}\label{step-2.2-plotting-your-first-spectrum}

Take the same \textbf{Plate}, \textbf{Fiber}, and \textbf{MJD} numbers
from earlier and enter them into the code below - make sure they match!

Then, run the cell.

\begin{verbatim}
<h3 class='alert-heading'>Helpful Reminder(s)</h3>
<ul>
    <li>Click anywhere inside of a cell to select it.</li>
    <li>Hit [Shift + Enter] to run a selected cell.</li>
</ul>
\end{verbatim}

    \begin{Verbatim}[commandchars=\\\{\}]
{\color{incolor}In [{\color{incolor} }]:} \PY{c+c1}{\PYZsh{} Fetch single spectrum \PYZhy{} Enter the same \PYZdq{}Plate\PYZdq{}, \PYZdq{}MJD\PYZdq{} and \PYZdq{}Fiber\PYZdq{} numbers here.}
        
        \PY{n}{plate} \PY{o}{=} \PY{l+m+mi}{300}
        \PY{n}{mjd} \PY{o}{=} \PY{l+m+mi}{51666}
        \PY{n}{fiber} \PY{o}{=} \PY{l+m+mi}{128}
        \PY{n}{spec} \PY{o}{=} \PY{n}{fetch\PYZus{}sdss\PYZus{}spectrum}\PY{p}{(}\PY{n}{plate}\PY{p}{,} \PY{n}{mjd}\PY{p}{,} \PY{n}{fiber}\PY{p}{)}
        
        \PY{c+c1}{\PYZsh{} now, just as before, we can plot the data}
        \PY{n}{figure}\PY{p}{(}\PY{p}{)}
        \PY{n}{plot}\PY{p}{(}\PY{n}{spec}\PY{o}{.}\PY{n}{wavelength}\PY{p}{(}\PY{p}{)}\PY{p}{,} \PY{n}{spec}\PY{o}{.}\PY{n}{spectrum}\PY{o}{/}\PY{n}{spec}\PY{o}{.}\PY{n}{spectrum}\PY{o}{.}\PY{n}{max}\PY{p}{(}\PY{p}{)}\PY{p}{,} \PY{n}{color}\PY{o}{=}\PY{l+s+s1}{\PYZsq{}}\PY{l+s+s1}{black}\PY{l+s+s1}{\PYZsq{}}\PY{p}{)}
        \PY{n}{xlabel}\PY{p}{(}\PY{l+s+s1}{\PYZsq{}}\PY{l+s+s1}{Wavelength}\PY{l+s+s1}{\PYZsq{}}\PY{p}{)}
        \PY{n}{ylabel}\PY{p}{(}\PY{l+s+s1}{\PYZsq{}}\PY{l+s+s1}{Brightness}\PY{l+s+s1}{\PYZsq{}}\PY{p}{)}
\end{Verbatim}


    That's the same spectrum as is on the webpage!

What you did was pull data from the web into your app. Now you can run
whatever code you like on the data!

This is a key element of data analysis.

Let's zoom in on the part of the spectrum we can see with our eyes -
"VIBGYOR" is about \(4000\) to \(7000\) Angstroms - or \(10^{-10}\)
meters (\(0.\) followed by nine zeros and then a \(1\) - really small!).

    \begin{Verbatim}[commandchars=\\\{\}]
{\color{incolor}In [{\color{incolor} }]:} \PY{c+c1}{\PYZsh{} Lets look at the range where our eyes can see.}
        
        \PY{n}{figure}\PY{p}{(}\PY{p}{)}
        \PY{n}{plot}\PY{p}{(}\PY{n}{spec}\PY{o}{.}\PY{n}{wavelength}\PY{p}{(}\PY{p}{)}\PY{p}{,} \PY{n}{spec}\PY{o}{.}\PY{n}{spectrum}\PY{o}{/}\PY{n}{spec}\PY{o}{.}\PY{n}{spectrum}\PY{o}{.}\PY{n}{max}\PY{p}{(}\PY{p}{)}\PY{p}{,} \PY{n}{color}\PY{o}{=}\PY{l+s+s1}{\PYZsq{}}\PY{l+s+s1}{black}\PY{l+s+s1}{\PYZsq{}}\PY{p}{)}
        \PY{n}{xlabel}\PY{p}{(}\PY{l+s+s1}{\PYZsq{}}\PY{l+s+s1}{Wavelength}\PY{l+s+s1}{\PYZsq{}}\PY{p}{)}
        \PY{n}{ylabel}\PY{p}{(}\PY{l+s+s1}{\PYZsq{}}\PY{l+s+s1}{Brightness}\PY{l+s+s1}{\PYZsq{}}\PY{p}{)}
        \PY{n}{xlim}\PY{p}{(}\PY{l+m+mi}{4000}\PY{p}{,}\PY{l+m+mi}{7000}\PY{p}{)}
        \PY{n}{ylim}\PY{p}{(}\PY{l+m+mi}{0}\PY{p}{,} \PY{l+m+mf}{1.2}\PY{p}{)}
\end{Verbatim}


    See those dips - those are the absorption lines we talked about! They
are the chemical fingerprint of a star.

This star has lines from Hydrogen and Helium - the first and second
elements on the periodic table, and the two most common elements in the
entire Universe.

You can also tell something about the star from its spectrum. It's
higher on the left (at lower wavelengths) than the right (at higher
wavelengths).

The low end of this wavelength range (\(4000\) Angstroms) is what our
eyes percieve as blue - or simply, this star will look blue! \_\_\_

    \subsection{Step 2.3: Plotting a Second
Spectrum}\label{step-2.3-plotting-a-second-spectrum}

Not all stars look the same. Pick another star from the page, get its
\textbf{Fiber}, \textbf{Plate}, and \textbf{MJD}, and let us plot that
too.

    \begin{Verbatim}[commandchars=\\\{\}]
{\color{incolor}In [{\color{incolor} }]:} \PY{c+c1}{\PYZsh{} Fetch a second spectrum.}
        
        \PY{n}{plate2} \PY{o}{=} \PY{l+m+mi}{273}
        \PY{n}{mjd2} \PY{o}{=} \PY{l+m+mi}{51957}
        \PY{n}{fiber2} \PY{o}{=} \PY{l+m+mi}{304}
        \PY{n}{spec2} \PY{o}{=} \PY{n}{fetch\PYZus{}sdss\PYZus{}spectrum}\PY{p}{(}\PY{n}{plate2}\PY{p}{,} \PY{n}{mjd2}\PY{p}{,} \PY{n}{fiber2}\PY{p}{)}
        \PY{n}{figure}\PY{p}{(}\PY{p}{)}
        \PY{n}{plot}\PY{p}{(}\PY{n}{spec}\PY{o}{.}\PY{n}{wavelength}\PY{p}{(}\PY{p}{)}\PY{p}{,} \PY{n}{spec}\PY{o}{.}\PY{n}{spectrum}\PY{o}{/}\PY{n}{spec}\PY{o}{.}\PY{n}{spectrum}\PY{o}{.}\PY{n}{max}\PY{p}{(}\PY{p}{)}\PY{p}{,} \PY{n}{color}\PY{o}{=}\PY{l+s+s1}{\PYZsq{}}\PY{l+s+s1}{black}\PY{l+s+s1}{\PYZsq{}}\PY{p}{)}
        \PY{n}{plot}\PY{p}{(}\PY{n}{spec2}\PY{o}{.}\PY{n}{wavelength}\PY{p}{(}\PY{p}{)}\PY{p}{,} \PY{n}{spec2}\PY{o}{.}\PY{n}{spectrum}\PY{o}{/}\PY{n}{spec2}\PY{o}{.}\PY{n}{spectrum}\PY{o}{.}\PY{n}{max}\PY{p}{(}\PY{p}{)}\PY{p}{,} \PY{n}{color}\PY{o}{=}\PY{l+s+s1}{\PYZsq{}}\PY{l+s+s1}{red}\PY{l+s+s1}{\PYZsq{}}\PY{p}{)}
        \PY{n}{xlabel}\PY{p}{(}\PY{l+s+s1}{\PYZsq{}}\PY{l+s+s1}{Wavelength}\PY{l+s+s1}{\PYZsq{}}\PY{p}{)}
        \PY{n}{ylabel}\PY{p}{(}\PY{l+s+s1}{\PYZsq{}}\PY{l+s+s1}{Brightness}\PY{l+s+s1}{\PYZsq{}}\PY{p}{)}
        \PY{n}{xlim}\PY{p}{(}\PY{l+m+mi}{4000}\PY{p}{,}\PY{l+m+mi}{7000}\PY{p}{)}
        \PY{n}{ylim}\PY{p}{(}\PY{l+m+mi}{0}\PY{p}{,} \PY{l+m+mf}{1.2}\PY{p}{)}
\end{Verbatim}


    Notice it has some of the same absorption lines in the same places as
the other star, but they are deeper - it has relatively more of that
element.

But that isn't all! This star, plotted in red, is lower on the left
(blue) side than the right (red) side - it'll look red!

\begin{quote}
Remember: From the H-R diagram, the color of stars is related to their
temperature - the redder star is cooler than the hot blue star!
\end{quote}

If you can figure out the color of stars, you can figure out their
temperature, their mass and their size.

\begin{verbatim}
<center>
    <br>
    <img src='http://astro.ib-host.net/pics/HR-diagram.jpg', width='600'>
    <br>
    <figcaption>
        <font color='grey'>
            <b>Figure 2:</b>
            The H-R diagram.
        </font>
    </figcaption>
</center>
\end{verbatim}

\begin{center}\rule{0.5\linewidth}{\linethickness}\end{center}

    \subsection{Step 2.4: Plotting Vega}\label{step-2.4-plotting-vega}

Plot the spectrum of Vega (in the middle of the upper plot), and see
what this star looks like.

    \begin{Verbatim}[commandchars=\\\{\}]
{\color{incolor}In [{\color{incolor} }]:} \PY{c+c1}{\PYZsh{} Fetch a third spectrum.}
        
        \PY{n}{spec3} \PY{o}{=} \PY{n}{fetch\PYZus{}vega\PYZus{}spectrum}\PY{p}{(}\PY{p}{)}
        \PY{n}{figure}\PY{p}{(}\PY{p}{)}
        \PY{n}{plot}\PY{p}{(}\PY{n}{spec3}\PY{p}{[}\PY{l+m+mi}{0}\PY{p}{]}\PY{p}{,} \PY{n}{spec3}\PY{p}{[}\PY{l+m+mi}{1}\PY{p}{]}\PY{o}{/}\PY{n}{spec3}\PY{p}{[}\PY{l+m+mi}{1}\PY{p}{]}\PY{o}{.}\PY{n}{max}\PY{p}{(}\PY{p}{)}\PY{p}{,} \PY{n}{color}\PY{o}{=}\PY{l+s+s1}{\PYZsq{}}\PY{l+s+s1}{blue}\PY{l+s+s1}{\PYZsq{}}\PY{p}{)}
        \PY{n}{plot}\PY{p}{(}\PY{n}{spec}\PY{o}{.}\PY{n}{wavelength}\PY{p}{(}\PY{p}{)}\PY{p}{,} \PY{n}{spec}\PY{o}{.}\PY{n}{spectrum}\PY{o}{/}\PY{n}{spec}\PY{o}{.}\PY{n}{spectrum}\PY{o}{.}\PY{n}{max}\PY{p}{(}\PY{p}{)}\PY{p}{,} \PY{n}{color}\PY{o}{=}\PY{l+s+s1}{\PYZsq{}}\PY{l+s+s1}{black}\PY{l+s+s1}{\PYZsq{}}\PY{p}{)}
        \PY{n}{plot}\PY{p}{(}\PY{n}{spec2}\PY{o}{.}\PY{n}{wavelength}\PY{p}{(}\PY{p}{)}\PY{p}{,} \PY{n}{spec2}\PY{o}{.}\PY{n}{spectrum}\PY{o}{/}\PY{n}{spec2}\PY{o}{.}\PY{n}{spectrum}\PY{o}{.}\PY{n}{max}\PY{p}{(}\PY{p}{)}\PY{p}{,} \PY{n}{color}\PY{o}{=}\PY{l+s+s1}{\PYZsq{}}\PY{l+s+s1}{red}\PY{l+s+s1}{\PYZsq{}}\PY{p}{)}
        \PY{n}{xlabel}\PY{p}{(}\PY{l+s+s1}{\PYZsq{}}\PY{l+s+s1}{Wavelength}\PY{l+s+s1}{\PYZsq{}}\PY{p}{)}
        \PY{n}{ylabel}\PY{p}{(}\PY{l+s+s1}{\PYZsq{}}\PY{l+s+s1}{Brightness}\PY{l+s+s1}{\PYZsq{}}\PY{p}{)}
        \PY{n}{xlim}\PY{p}{(}\PY{l+m+mi}{4000}\PY{p}{,}\PY{l+m+mi}{7000}\PY{p}{)}
        \PY{n}{ylim}\PY{p}{(}\PY{l+m+mi}{0}\PY{p}{,} \PY{l+m+mf}{1.2}\PY{p}{)}
\end{Verbatim}


    This third star plotted in blue, is Vega - one of the brightest stars in
the sky.

It's one of the three stars that make up the Summer Triangle.

\begin{verbatim}
<center>
    <br>
    <img src='http://astropixels.com/milkyway/wide/images/MilkyWay12-501w.jpg', width='600'>
    <br>
    <figcaption>
        <font color='grey'>
            <b>Figure 3:</b>
            The three bright stars of the Summer Triangle with our galaxy, the Milky Way, behind. Each of those specs of light is a star! There are some hundred billion stars in our Galaxy alone!
        </font>
    </figcaption>
</center>
\end{verbatim}

Even here you can see that stars have a range of colors - or really
temperatures.

Remember that the stars need fuel to resist the crushing pull of
gravity. While stars burn Hydrogen and produce Helium in their core,
they are said to be on \textbf{The Main Sequence}.

You could scroll up and look at that figure again... OR we can just plot
the main sequence from data ourselves! \_\_\_

    \section{Activity 3: Plotting the Main
Sequence}\label{activity-3-plotting-the-main-sequence}

Now that we've seen that spectra of stars are different, lets compare
the brightness and colors of a lot of different stars. \_\_\_

    \subsection{Step 3.1: Entering the Data from Some
Stars}\label{step-3.1-entering-the-data-from-some-stars}

This is just like Section \ref{activity-1-graphing-some-data}, except
instead of calling it "x" and "y", I'm calling it "color" and
"brightness".

I've already entered the data below - all you have to do is run the
cell.

\begin{verbatim}
<h3 class='alert-heading'>Helpful Reminder(s)</h3>
<ul>
    <li>Click anywhere inside of a cell to select it.</li>
    <li>Hit [Shift + Enter] to run a selected cell.</li>
</ul>
\end{verbatim}

    \begin{Verbatim}[commandchars=\\\{\}]
{\color{incolor}In [{\color{incolor} }]:} \PY{c+c1}{\PYZsh{} First, lets take some data that I\PYZsq{}ve entered directly from the SDSS}
        \PY{n}{star} \PY{o}{=} \PY{p}{[}\PY{l+s+s1}{\PYZsq{}}\PY{l+s+s1}{Sun}\PY{l+s+s1}{\PYZsq{}}\PY{p}{,} \PY{l+s+s1}{\PYZsq{}}\PY{l+s+s1}{Sirius}\PY{l+s+s1}{\PYZsq{}}\PY{p}{,} \PY{l+s+s1}{\PYZsq{}}\PY{l+s+s1}{Canopus}\PY{l+s+s1}{\PYZsq{}}\PY{p}{,} \PY{l+s+s1}{\PYZsq{}}\PY{l+s+s1}{Arcturus}\PY{l+s+s1}{\PYZsq{}}\PY{p}{,} \PY{l+s+s1}{\PYZsq{}}\PY{l+s+s1}{AlphaCen}\PY{l+s+s1}{\PYZsq{}}\PY{p}{,} \PY{l+s+s1}{\PYZsq{}}\PY{l+s+s1}{Vega}\PY{l+s+s1}{\PYZsq{}}\PY{p}{,}\PY{l+s+s1}{\PYZsq{}}\PY{l+s+s1}{Capella}\PY{l+s+s1}{\PYZsq{}}\PY{p}{,} \PY{l+s+s1}{\PYZsq{}}\PY{l+s+s1}{Rigel}\PY{l+s+s1}{\PYZsq{}}\PY{p}{,} \PY{l+s+s1}{\PYZsq{}}\PY{l+s+s1}{Procyon}\PY{l+s+s1}{\PYZsq{}}\PY{p}{,} \PY{l+s+s1}{\PYZsq{}}\PY{l+s+s1}{Betelgeuse}\PY{l+s+s1}{\PYZsq{}}\PY{p}{,} \PY{l+s+s1}{\PYZsq{}}\PY{l+s+s1}{Achernar}\PY{l+s+s1}{\PYZsq{}}\PY{p}{,}\PY{l+s+s1}{\PYZsq{}}\PY{l+s+s1}{Hadar}\PY{l+s+s1}{\PYZsq{}}\PY{p}{,} \PY{l+s+s1}{\PYZsq{}}\PY{l+s+s1}{Acrux}\PY{l+s+s1}{\PYZsq{}}\PY{p}{,} \PY{l+s+s1}{\PYZsq{}}\PY{l+s+s1}{Altair}\PY{l+s+s1}{\PYZsq{}}\PY{p}{,} \PY{l+s+s1}{\PYZsq{}}\PY{l+s+s1}{Aldebaran}\PY{l+s+s1}{\PYZsq{}}\PY{p}{,} \PY{l+s+s1}{\PYZsq{}}\PY{l+s+s1}{Antares}\PY{l+s+s1}{\PYZsq{}}\PY{p}{,} \PY{l+s+s1}{\PYZsq{}}\PY{l+s+s1}{Spica}\PY{l+s+s1}{\PYZsq{}}\PY{p}{,} \PY{l+s+s1}{\PYZsq{}}\PY{l+s+s1}{Pollux}\PY{l+s+s1}{\PYZsq{}}\PY{p}{,} \PY{l+s+s1}{\PYZsq{}}\PY{l+s+s1}{Formalhaut}\PY{l+s+s1}{\PYZsq{}}\PY{p}{,} \PY{l+s+s1}{\PYZsq{}}\PY{l+s+s1}{Becrux}\PY{l+s+s1}{\PYZsq{}}\PY{p}{,} \PY{l+s+s1}{\PYZsq{}}\PY{l+s+s1}{Deneb}\PY{l+s+s1}{\PYZsq{}}\PY{p}{,} \PY{l+s+s1}{\PYZsq{}}\PY{l+s+s1}{Regulus}\PY{l+s+s1}{\PYZsq{}}\PY{p}{,} \PY{l+s+s1}{\PYZsq{}}\PY{l+s+s1}{Adhara}\PY{l+s+s1}{\PYZsq{}}\PY{p}{,}\PY{l+s+s1}{\PYZsq{}}\PY{l+s+s1}{Shaula}\PY{l+s+s1}{\PYZsq{}}\PY{p}{,} \PY{l+s+s1}{\PYZsq{}}\PY{l+s+s1}{Gacrux}\PY{l+s+s1}{\PYZsq{}}\PY{p}{,} \PY{l+s+s1}{\PYZsq{}}\PY{l+s+s1}{Castor}\PY{l+s+s1}{\PYZsq{}}\PY{p}{]}
        \PY{n}{apparent\PYZus{}brightness} \PY{o}{=} \PY{p}{[}\PY{o}{\PYZhy{}}\PY{l+m+mf}{26.8}\PY{p}{,} \PY{o}{\PYZhy{}}\PY{l+m+mf}{1.46}\PY{p}{,} \PY{o}{\PYZhy{}}\PY{l+m+mf}{0.72}\PY{p}{,} \PY{o}{\PYZhy{}}\PY{l+m+mf}{0.04}\PY{p}{,} \PY{o}{\PYZhy{}}\PY{l+m+mf}{0.01}\PY{p}{,} \PY{l+m+mf}{0.0}\PY{p}{,} \PY{l+m+mf}{0.08}\PY{p}{,} \PY{l+m+mf}{0.12}\PY{p}{,} \PY{l+m+mf}{0.38}\PY{p}{,} \PY{l+m+mf}{0.41}\PY{p}{,} \PY{l+m+mf}{0.46}\PY{p}{,} \PY{l+m+mf}{0.63}\PY{p}{,} \PY{l+m+mf}{0.76}\PY{p}{,} \PY{l+m+mf}{0.77}\PY{p}{,} \PY{l+m+mf}{0.85}\PY{p}{,} \PY{l+m+mf}{0.92}\PY{p}{,} \PY{l+m+mf}{1.0}\PY{p}{,} \PY{l+m+mf}{1.14}\PY{p}{,} \PY{l+m+mf}{1.16}\PY{p}{,} \PY{l+m+mf}{1.2}\PY{p}{,} \PY{l+m+mf}{1.25}\PY{p}{,} \PY{l+m+mf}{1.35}\PY{p}{,} \PY{l+m+mf}{1.5}\PY{p}{,} \PY{l+m+mf}{1.6}\PY{p}{,} \PY{l+m+mf}{1.63}\PY{p}{,} \PY{l+m+mf}{1.98}\PY{p}{]}
        \PY{n}{brightness} \PY{o}{=} \PY{p}{[}\PY{l+m+mf}{4.8}\PY{p}{,} \PY{l+m+mf}{1.4}\PY{p}{,} \PY{o}{\PYZhy{}}\PY{l+m+mf}{2.5}\PY{p}{,} \PY{l+m+mf}{0.2}\PY{p}{,} \PY{l+m+mf}{4.4}\PY{p}{,} \PY{l+m+mf}{0.6}\PY{p}{,} \PY{l+m+mf}{0.4}\PY{p}{,} \PY{o}{\PYZhy{}}\PY{l+m+mf}{8.1}\PY{p}{,} \PY{l+m+mf}{2.6}\PY{p}{,} \PY{o}{\PYZhy{}}\PY{l+m+mf}{7.2}\PY{p}{,} \PY{o}{\PYZhy{}}\PY{l+m+mf}{1.3}\PY{p}{,} \PY{o}{\PYZhy{}}\PY{l+m+mf}{4.4}\PY{p}{,} \PY{o}{\PYZhy{}}\PY{l+m+mf}{4.6}\PY{p}{,} \PY{l+m+mf}{2.3}\PY{p}{,} \PY{o}{\PYZhy{}}\PY{l+m+mf}{0.3}\PY{p}{,} \PY{o}{\PYZhy{}}\PY{l+m+mf}{5.2}\PY{p}{,} \PY{o}{\PYZhy{}}\PY{l+m+mf}{3.2}\PY{p}{,} \PY{l+m+mf}{0.7}\PY{p}{,} \PY{l+m+mf}{2.0}\PY{p}{,} \PY{o}{\PYZhy{}}\PY{l+m+mf}{4.7}\PY{p}{,} \PY{o}{\PYZhy{}}\PY{l+m+mf}{7.2}\PY{p}{,} \PY{o}{\PYZhy{}}\PY{l+m+mf}{0.3}\PY{p}{,} \PY{o}{\PYZhy{}}\PY{l+m+mf}{4.8}\PY{p}{,} \PY{o}{\PYZhy{}}\PY{l+m+mf}{3.5}\PY{p}{,} \PY{o}{\PYZhy{}}\PY{l+m+mf}{1.2}\PY{p}{,} \PY{l+m+mf}{0.5}\PY{p}{]}
        \PY{n}{color} \PY{o}{=} \PY{p}{[}\PY{l+m+mf}{0.63}\PY{p}{,} \PY{l+m+mf}{0.0}\PY{p}{,} \PY{l+m+mf}{0.15}\PY{p}{,} \PY{l+m+mf}{1.23}\PY{p}{,} \PY{l+m+mf}{0.71}\PY{p}{,} \PY{l+m+mf}{0.0}\PY{p}{,} \PY{l+m+mf}{0.08}\PY{p}{,} \PY{o}{\PYZhy{}}\PY{l+m+mf}{0.03}\PY{p}{,} \PY{l+m+mf}{0.42}\PY{p}{,} \PY{l+m+mf}{1.85}\PY{p}{,} \PY{o}{\PYZhy{}}\PY{l+m+mf}{0.16}\PY{p}{,} \PY{o}{\PYZhy{}}\PY{l+m+mf}{0.23}\PY{p}{,} \PY{o}{\PYZhy{}}\PY{l+m+mf}{0.24}\PY{p}{,} \PY{l+m+mf}{0.22}\PY{p}{,} \PY{l+m+mf}{1.54}\PY{p}{,} \PY{l+m+mf}{1.83}\PY{p}{,} \PY{o}{\PYZhy{}}\PY{l+m+mf}{0.23}\PY{p}{,} \PY{l+m+mf}{1.0}\PY{p}{,} \PY{l+m+mf}{0.09}\PY{p}{,} \PY{o}{\PYZhy{}}\PY{l+m+mf}{0.23}\PY{p}{,} \PY{l+m+mf}{0.09}\PY{p}{,} \PY{o}{\PYZhy{}}\PY{l+m+mf}{0.11}\PY{p}{,} \PY{o}{\PYZhy{}}\PY{l+m+mf}{0.21}\PY{p}{,} \PY{o}{\PYZhy{}}\PY{l+m+mf}{0.22}\PY{p}{,} \PY{l+m+mf}{1.59}\PY{p}{,} \PY{l+m+mf}{0.03}\PY{p}{]}
\end{Verbatim}


    \begin{center}\rule{0.5\linewidth}{\linethickness}\end{center}

    \subsection{Step 3.2: Plot the Star
Data}\label{step-3.2-plot-the-star-data}

Lets plot the data we entered in the last two cells with the scatter
command we used in Section \ref{activity-1-graphing-some-data}.

    \begin{Verbatim}[commandchars=\\\{\}]
{\color{incolor}In [{\color{incolor} }]:} \PY{c+c1}{\PYZsh{} We\PYZsq{}ll use the size property \PYZsq{}s\PYZsq{} to scale the points by the brightness.}
        
        \PY{n}{figure}\PY{p}{(}\PY{p}{)}
        \PY{n}{cmap} \PY{o}{=} \PY{n}{cm}\PY{o}{.}\PY{n}{ScalarMappable}\PY{p}{(}\PY{n}{cmap}\PY{o}{=}\PY{l+s+s1}{\PYZsq{}}\PY{l+s+s1}{jet}\PY{l+s+s1}{\PYZsq{}}\PY{p}{)}
        \PY{n}{colors}\PY{o}{=}\PY{n}{cmap}\PY{o}{.}\PY{n}{to\PYZus{}rgba}\PY{p}{(}\PY{n}{arctan}\PY{p}{(}\PY{n}{array}\PY{p}{(}\PY{n}{color}\PY{p}{)}\PY{p}{)}\PY{p}{)}
        \PY{n}{scatter}\PY{p}{(}\PY{n}{color}\PY{p}{,}\PY{n}{brightness}\PY{p}{,} \PY{n}{marker}\PY{o}{=}\PY{l+s+s1}{\PYZsq{}}\PY{l+s+s1}{*}\PY{l+s+s1}{\PYZsq{}}\PY{p}{,} \PY{n}{color}\PY{o}{=}\PY{n}{colors}\PY{p}{,} \PY{n}{s}\PY{o}{=}\PY{l+m+mi}{10} \PY{o}{+} \PY{l+m+mi}{10}\PY{o}{*}\PY{o}{*}\PY{p}{(}\PY{o}{\PYZhy{}}\PY{l+m+mf}{0.4}\PY{o}{*}\PY{n}{array}\PY{p}{(}\PY{n}{brightness}\PY{p}{)}\PY{p}{)}\PY{p}{)}
        
        \PY{c+c1}{\PYZsh{}\PYZsh{} if you want to see which star is which, remove the \PYZsq{}\PYZsh{}\PYZsq{} in front of the next two lines}
        
        \PY{k}{for} \PY{n}{i}\PY{p}{,} \PY{n}{name} \PY{o+ow}{in} \PY{n+nb}{enumerate}\PY{p}{(}\PY{n}{star}\PY{p}{)}\PY{p}{:}
            \PY{n}{annotate}\PY{p}{(}\PY{n}{name}\PY{p}{,}\PY{p}{(}\PY{n}{color}\PY{p}{[}\PY{n}{i}\PY{p}{]}\PY{p}{,}\PY{n}{brightness}\PY{p}{[}\PY{n}{i}\PY{p}{]}\PY{p}{)}\PY{p}{)}
        
        \PY{n}{xlabel}\PY{p}{(}\PY{l+s+s1}{\PYZsq{}}\PY{l+s+s1}{Color}\PY{l+s+s1}{\PYZsq{}}\PY{p}{)}
        \PY{n}{ylabel}\PY{p}{(}\PY{l+s+s1}{\PYZsq{}}\PY{l+s+s1}{Brightness}\PY{l+s+s1}{\PYZsq{}}\PY{p}{)}
        \PY{n}{suptitle}\PY{p}{(}\PY{l+s+s1}{\PYZsq{}}\PY{l+s+s1}{Main Sequence of the Brightest Stars}\PY{l+s+s1}{\PYZsq{}}\PY{p}{)}
        
        \PY{c+c1}{\PYZsh{} Brightness is a little weird \PYZhy{} remember that smaller numbers mean that the star is brighter }
        \PY{c+c1}{\PYZsh{} It\PYZsq{}s a little like the Top 100 charts \PYZhy{} a ranked list}
        \PY{c+c1}{\PYZsh{} With the smaller the number being the higher rank}
        \PY{n}{plt}\PY{o}{.}\PY{n}{gca}\PY{p}{(}\PY{p}{)}\PY{o}{.}\PY{n}{invert\PYZus{}xaxis}\PY{p}{(}\PY{p}{)}
        \PY{n}{plt}\PY{o}{.}\PY{n}{gca}\PY{p}{(}\PY{p}{)}\PY{o}{.}\PY{n}{invert\PYZus{}yaxis}\PY{p}{(}\PY{p}{)}
\end{Verbatim}


    \begin{center}\rule{0.5\linewidth}{\linethickness}\end{center}

    \subsection{Step 3.3: Get a Sense for the Sizes and Colors of These
Stars}\label{step-3.3-get-a-sense-for-the-sizes-and-colors-of-these-stars}

The stars on the left hand side like Betelgeuse and Antares, are red
giant stars nearing the end of their lives.

\begin{verbatim}
<center>
    <br>
    <img src='https://upload.wikimedia.org/wikipedia/commons/4/47/Orion_Head_to_Toe.jpg', width='600'>
    <br>
    <figcaption>
        <font color='grey'>
            <b>Figure 4:</b>
            Betelgeuse is in the constellation Orion - and you can easily see it rise over Tucson at night. If you look carefully, you'll be able to tell it's a different color with just your own eyes.
        </font>
    </figcaption>
</center>
\end{verbatim}

Remember, different stars also have different sizes. Compare the size of
Betelgeuse to the Sun by scrolling from left to right on the image
below.

    \begin{Verbatim}[commandchars=\\\{\}]
{\color{incolor}In [{\color{incolor} }]:} \PY{n}{display}\PY{p}{(}\PY{n}{Image}\PY{p}{(}\PY{l+s+s1}{\PYZsq{}}\PY{l+s+s1}{star\PYZus{}sizes\PYZus{}small.jpg}\PY{l+s+s1}{\PYZsq{}}\PY{p}{,} \PY{n}{width}\PY{o}{=}\PY{l+m+mi}{4000}\PY{p}{,} \PY{n}{unconfined}\PY{o}{=}\PY{k+kc}{True}\PY{p}{)}\PY{p}{)}
\end{Verbatim}


    

    \section{Activity 4: Looking at the Life Cycles of
Stars}\label{activity-4-looking-at-the-life-cycles-of-stars}

We've taken a look at different stellar spectra, and put them on context
with each other with the Main Sequence. Stars may look like they are
always the same to us, but they're always changing.

As a star runs out of fuel, it is no longer able to withstand the force
of gravity and begins to die. Stars die in several different ways. Some
end their lives as white dwarfs, slowly cooling in space. The most
massive stars end their lives as nuetron stars, and black holes. Many
stars however, die by exploding spectacularly. These are supernovae!

Use the \href{http://starinabox.lco.global/}{Star in a box} to see how
stars with different masses live and die!


    % Add a bibliography block to the postdoc
    
    
    
    \end{document}
